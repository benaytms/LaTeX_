\documentclass[12pt, a4paper]{article}
\usepackage{graphicx}
\usepackage{braket}
\usepackage{enumitem}
\usepackage{indentfirst}
\usepackage[margin=1.5cm]{geometry}


\begin{document}

\begin{center}
    
    {\LARGE
        Análise Orientada a Objetos - Aula 01\\
    }
    {\Large
        Benay T. da L. de Carvalho - Universidade Norte do Paraná\\
        RA: 3681235802
    }

\end{center}

\vspace{1cm}

\large
\noindent
\textbf{Paradigma Orientado a Objetos}

No desenvolvimento de Software há diversos paradigmas - que são conjuntos de preceitos e métodos que servem para organizar e nos aproximar dos problemas com diferentes formas.
Os mais comuns paradigmas são: \textbf{Imperativo, Procedural, Funcional, Declarativo e Orientado a Objetos.}

\vspace*{0.5cm}

{\setlength{\parindent}{0cm}
    \textbf{Pontos Principais da Orientação a Objetos}
}

\begin{enumerate}[label*=\textbf{\arabic*}.]
    
    \item Gira em torno de classes, objetos, métodos e atributos.
    
    \item Classes são as classificações dos objetos. A classe pessoa pode ter o objeto
funcionário que terá os atributos: Nome, idade, gênero, número de identificação, etc. E os métodos
são as ações que o objeto pode fazer, nessa situação por exemplo, o funcionário pode "bater o ponto".
    
    \item Utiliza alguns tipos de relacionamentos entre classes, exemplos:
    
    \begin{enumerate}[label*=\textbf{\arabic*}.]

        \item \textbf{Herança:} A herança funciona de forma a passar certas características (atributos)
        de uma classe (Super) para outra (Sub). Exemplificando, a classe liquído (Super)
        irá herdar para a classe suco (Sub) e para classe refrigerante (Sub), a consistência.
        
        \item \textbf{Polimorfismo:} Digamos que ocorre uma herança entre classes e duas ou mais classes filhas
        herdam os mesmos elementos de uma classe pai, porém utilizam esses elementos de maneiras diferentes,
        isso é chamado de polimorfismo. Em suma, é quando diferentes classes têm métodos com mesmo nome mas funções
        diferentes.
        
        \item \textbf{Encapsulamento: } É a limitação dos atributos e/ou métodos de uma classe pai que são herdados
        à uma classe filha. Por exemplo, digamos que iremos herdar alguns atributos do CEO para o Gerente de um 
        estabelecimento e para evitar passar algumas permissões que apenas o CEO deve guardar, iremos limitar essa herança para ser herdado apenas 
        o que o Gerente deve ter.

    \end{enumerate}

\end{enumerate}

\newpage

\vspace*{1cm}

\noindent
\textbf{Unified Modelling Language}

Ou UML é uma linguagem de modelagem que utiliza da Orientação a Objetos.
É compatível com desenvolvimento de Software desde os requisitos até a implantação.

\noindent
\textbf{\\Objetivos da UML}
\begin{itemize}
    \item Modelar em diferentes linguagens e situações.
    \item Padrão para o desenvolvimento de software.
    \item Ser simples.
    \item Funcionar como forma de documentar para futuras alterações do sistema.
\end{itemize}

Os diagramas em UML possuem níveis de abstração, isso dita o quão complexo será o diagrama.
Quanto maior o nível de abstração mais fácil será o entendimento.

\noindent
\textbf{\\Níveis de abstração: }
\begin{itemize}
    \item \textbf{Alto:} Ser o mais claro e simples possivel, serve para apresentar ao cliente
    \item \textbf{Médio:} Guiar o desenvolvimento apresentado, 
    sem detalhar muito as classes, os objetos e as interações.
    \item \textbf{Baixo:} Demonstrar como será o sistema de forma precisa, 
    especificando cada interação e cada método.
\end{itemize}

\noindent
\textbf{\\Tipos de Diagramas}
\begin{enumerate}[label*=\textbf{\arabic*}.]

    \item \textbf{Diagramas Estruturais}
        \begin{enumerate}[label*=\textbf{\arabic*}.]
            \item \textbf{Classes:} Representa a estrutura e relação entre as classes
            que moldam os objetos. \textbf{Elementos:} Atributo, Operação e Associação.

            \item \textbf{Objetos:} Utiliza uma modelagem parecida com o diagrama de classes
            a diferença é que mostra os objetos que foram instanciados.

            \item \textbf{Pacotes:} Os pacotes são agrupamentos lógicos formado por pedaços
            do sistema.  Os pacotes se relacionam com outros pacotes através de uma relação de dependência.
            Este diagrama é muito utilizado para ilustrar a arquitetura de um sistema.

            \item \textbf{Estruturas Compostas:} Destina-se a descrição dos relacionamentos entre os elementos. 
            Utilizado para descrever a colaboração interna de classes, interfaces ou 
            componentes para especificar uma funcionalidade. \textbf{Elementos:} Colaboração, parte, port, papéis.

            \item \textbf{Componentes:} Ilustra como as classes deverão se organizar e relacionar. 
            Pode-se explicitar a qual classe cada componente pertence. Utiliza o diagrama de classes como base.

            \item \textbf{Implantações:} Descreve os componentes de hardware e Software
            e suas interações com outros elementos. Representa a configuração e arquitetura
            do sistema. \textbf{Elementos:} Nó (Peça física na qual o sistema será implantado), Artefatos(qualquer pedaço físico de informação instalado em um nó).
            
            \item \textbf{Perfis.}
        
        \end{enumerate}

    \item \textbf{Diagramas Comportamentais}
        \begin{enumerate}[label*=\textbf{\arabic*}.]

            \item \textbf{Casos de Uso:} Descreve a funcionalidade proposta para um sistema. 
            Representa as interações entre objetos e seus métodos de forma graficamente simples.
            Muito utilizado no levantamento de requisitos. \textbf{Elementos:} Ator, caso de uso e conexões(Include, extend, associação, generalização).
            
            \item \textbf{Atividade:} Essencialmente é um fluxograma das atividades empregadas.
            Tendo um ênfase no fluxo de controle entre atividades. Diferentemente do caso de uso, no diagrama de atividades
            não há atores e o foco é somente nas atividades/métodos e suas conexões. \textbf{Os principais elementos são:}
            Atividades, sub-atividades, transição, decisão, ação, bifurcação, raia e envio/recepção de sinal. 
            
            \item \textbf{Máquina de Estados:} Representa o estado ou situação que um objeto pode se encontrar.
            Foca nas transições de estados de um objeto. \textbf{Elementos:} Estado (Condição de um objeto na qual ele 
            executa certas atividades e aguarda certos eventos), Transição (Relação entre dois estados), Condição (Causa necessária para ocorrência de um estado).
            
        \end{enumerate}
    
    \item \textbf{Diagramas de Interação (Fazem parte dos diagramas comportamentais)}
        \begin{enumerate}[label*=\textbf{\arabic*}.]

            \item \textbf{Sequência:} Segue o diagrama de casos de uso, porém com o adicional de quê mostra
            as mensagens passadas entre objeto. Descreve a maneira como objetos colaboram em algum comportamento.
            \textbf{Elementos:} Linha de vida (Define o ponto de posição dos atores e objetos), Fragmento (Interações possiveis de um objeto para o outro).
            
            \item \textbf{Colaboração:} Sendo isomórfico ao diagrama de sequência, a diferença do diagrama de colaboração
            é focar na organização dos objetos, enquanto o de sequência tem o foco na relação entre objetos e no seu tempo.
            
            \item \textbf{Tempo:} Sendo semelhante ao dois anteriores, 
            no diagrama de tempo há marcações de tempo para cada evento.
            
            \item \textbf{Interação:} Diagramas de interação combinam diagramas de atividade com diagrama de sequência.
            É um tipo de diagrama de atividade que representa o envio ou o recebimento de dados entre um ator e um caso de uso.
            
        \end{enumerate}

\end{enumerate}

\newpage
\vspace*{1cm}
\noindent
\textbf{Ciclo de incrementação do Processo Unificado}

O processo unificado funciona fazendo constantes incrementações no Software
até chegar no resultado desejado. Esse ciclo pode ser representado por 4 fases, veja à seguir:
\textbf{Concepção, Elaboração, Construção, Implementação.} Dentro das fase há sub-níveis, que são os seguintes:
\textbf{Comunicação, Planejamento, Modelagem, Construção, Entrega e Incremento.}

A fase de \textbf{Concepção} engloba: Comunicação e planejamento. \textbf{Elaboração} engloba: Planejamento e modelagem.
\textbf{Construção} engloba: Modelagem e Construção. \textbf{Implementação} engloba: Construção e entrega.
e por fim, é feito o Incremento no software.

Diante de cada fase, é utilizado certos diagramas. E a cada fase que se passa, os diagramas da fase anterior
ainda são utilizados, por isso, não irei escrever os mesmos a cada fase mas que fique compreendido.

\noindent
\textbf{\\Fase de Concepção}
\begin{itemize}{\bfseries}
    \item Diagrama de Casos de Uso.
    \item Diagrama de Sequência.
    \item Diagrama de Colaboração.
    \item Diagrama de Atividades.
    \item Diagrama de Máquina de Estados.
\end{itemize}

\noindent
\textbf{\\Fase de Elaboração}
\begin{itemize}{\bfseries}
    \item Diagrama de Classes.
\end{itemize}

\noindent
\textbf{\\Fase de Construção}
\begin{itemize}
    \item Diagrama de Implantação.
\end{itemize}

\noindent
\textbf{\\Fase de Implementação}
\begin{itemize}
    \item Diagrama de Componentes.
\end{itemize}

\newpage
\noindent
\textbf{\\Mecanismos Comuns da UML}

\begin{enumerate}[label*=\textbf{\arabic*}.]
    \item \textbf{Especificação:} É a parte de detalhamento sobre os métodos, descrição dos casos de uso, nome dos objetos, etc.
    Em suma, é a parte escrita dos diagramas.
    \item \textbf{Adorno:} É a parte Gráfica-Visual dos diagramas, como os atores e os casos de uso por exemplo.
    \item \textbf{Divisões.}
    \item \textbf{Extensões.}
\end{enumerate}

\vfill
\begin{center}
    Curitiba\\
    2023
\end{center}

\end{document}