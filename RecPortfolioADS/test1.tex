\documentclass[12pt, a4paper]{article}
\usepackage{graphicx}
\usepackage{times}
\usepackage{braket}
\usepackage{enumitem}
\usepackage{indentfirst}
\usepackage[left=3cm, right=2cm, bottom=2cm, top=3cm]{geometry}

\begin{document}
\thispagestyle{empty}

\begin{center}
\includegraphics[scale=0.5]{unoparlogo2desnivel.png} \\
{\LARGE
\textbf{Universidade Norte do Paraná} \\
\textbf{Polo Sítio Cercado}\\
}\vspace{3cm}
{\Large
BENAY TOMAS DA LUZ DE CARVALHO - RA: 3681235801\\
\vspace{4cm}
\textbf{Análise e modelagem de sistemas - Engenharia de software}\\
Recuperação do relatório da aula prática\\
\vspace{9cm}
CURITIBA/PR\\
2023
}
\end{center}

\newpage
\thispagestyle{empty}
\begin{center}
{\Large
BENAY TOMAS DA LUZ DE CARVALHO\\
\vspace{5cm}
\textbf{Análise e modelagem de sistemas - Engenharia de software}\\
Recuperação do relatório da aula prática\\
\vspace{3cm}
\hspace*{3.3cm}Trabalho de portfólio apresentado como requisito
\hspace*{3.3cm}recuperativo parcial para obtenção dos pontos da
\hspace*{3.6cm}média semestral do curso Engenharia de Software\\
\hspace*{6.5cm}Orientadora: Jessica Fernandes Lopes\\
\vspace{9cm}
CURITIBA/PR\\
2023
}

\end{center}

\newpage
\begin{center}
{\Large
   \textbf{Sumário}
}
\end{center}

\renewcommand{\contentsname}{}
\tableofcontents


\newpage
\section{Introdução}

Este relatório tem como intuito demonstrar a utilização da linguagem
de modelagem UML (Unified Modelling Language) através do tipo
classificador – Caso de Uso, ou Use Case.

Esse é o padrão de diagrama mais utilizado dentro da linguagem e tem
como função principal proporcionar uma visão bem clara da série de ações a
serem feitas e conexões que um usuário tem com outras entidades em um
sistema (Físico ou Software), definindo de maneira sucinta e gráfica como
certos passos ocorrem a fim de entregar ao cliente e/ou usuário seu objetivo.

Neste projeto, o objetivo será desenvolver um diagrama de Caso de
Uso para um sistema bancário. Tendo um cliente, um funcionário e um caixa
eletrônico, sendo que cada um deles tem uma série de ações. Por exemplo, o
cliente pode abrir e encerrar sua conta, pode emitir seu saldo, pode depositar
dinheiro na conta, entre outras. Já o funcionário e o caixa eletrônico irão
executar essas ações que o cliente deseja.

\newpage
\vspace{1cm}
\section{Desenvolvimento}

A Unified Modelling Language, ou Linguagem de Modelagem
Unificada em português, tem algumas características muito importantes em
na modelagem de sistemas, portanto acho válido destacar brevemente:
\begin{itemize}
   \item Padronização – Existe uma vasta quantidade de tipos de Diagramas
com cada um tendo suas regras, notações e símbolos. O que permite o
analista de Software ter maior liberdade na hora de desenvolver um padrão
sistemático para sua equipe.

   \item Visualização – Por ser uma linguagem de modelagem, ter uma boa
visualização é essencial e isso é um dos aspectos mais fortes por parte da
UML. Exemplo disso é o próprio Caso de Uso que contém diferentes tipos
de conectivos e o uso de atores, o que torna a experiência de leitura de um
diagrama Use Case mais fácil e satisfatória.

   \item Documentação – Tendo uma visualização boa consequentemente faz
com que a \\documentação seja boa, sendo assim em caso de manutenção ou
mudança por parte do analista de Software, o fácil entendimento ajudará na
adaptação diante as mudanças.
\end{itemize}

Neste relatório em questão, como iremos utilizar Caso de Uso farei
também uma breve nota explicativa de cada elemento básico desse modelo
em específico, a seguir em 2.1 Métodos.

\vspace{1cm}
\subsection{Métodos}

Seguindo o Roteiro de Aula Prática para Análise e Modelagem de
Sistemas, os objetivos eram os seguintes:
\begin{enumerate}
   \item Criar um diagrama para um sistema bancário com intuito de
organizar as ações dos atores envolvidos, assim como esclarecido na
Introdução (p. 4).
   \item Contanto que suprisse certas especificações, o cliente poderia ter as
seguintes ações: abrir/encerrar conta, depositar/sacar dinheiro, emitir
saldo/extrato.
5
   \item O funcionário ajudaria o cliente, abrindo e encerrando a conta,
sendo que para abrir a conta o cliente deve especificar o tipo de conta que
deseja (Especial ou Poupança) e para fechar o cliente deve estar com o
saldo zerado.
   \item O caixa eletrônico realizaria as ações de depositar ou sacar
dinheiro e emitir saldo ou extrato.
\end{enumerate}

Para a elaboração do diagrama segui o seguinte processo: Separação
e seleção dos atores e criação de todas as ações possíveis. Em seguida
posicionei os elementos onde achei que ficaria melhor, seguindo uma
padronização para deixar tudo alinhado e por fim coloquei uma anotação
onde achei que seria necessário.
Na página seguinte, como citei anteriormente no Desenvolvimento, irei explicar os
elementos utilizados.

\newpage
\vspace{0.75cm}

{\Large Ator}
\begin{figure}[h]
   \centering
   \includegraphics[width=3cm, height=2cm]{ActorUseCase.png}
   \caption{Astah, 2019}
\end{figure}

   O ator é um elemento que representa as entidades que irão realizar e
interagir com os Casos de Uso. Podem ser pessoas, máquinas,
computadores, etc.

\vspace{0.75cm}

{\Large Caso de Uso}
\begin{figure}[h]
   \centering
   \includegraphics[width=7cm, height=4cm]{UseCase.png}
   \caption{Visual Paradigm, 2023}
\end{figure}

   O elemento Caso de Uso se trata de representar cada uma das ações,
seus nomes, suas propriedades e especificações. Ele é a raiz do diagrama por
isso recebe o nome do próprio modelo.

\vspace{0.75cm}

{\Large Conectivos}

\vspace{0.75cm}

   Os conectivos têm como objetivo representar as ligações que os elementos têm entre si.\\ 
Há uma variedade de conectivos, porém, estarei mostrando apenas os 4 principais, veja a seguir:

\vspace{0.75cm}

{\Large Associação}
\begin{figure}[h]
   \centering
   \includegraphics[width=4cm, height=3cm]{Association.png}
   \caption{Visual Paradigm, 2023}
\end{figure}

   A Associação tem o intuito de fazer a ligação entre atores e Casos de
Uso, ela é o que conecta o cliente/usuário com as devidas funcionalidades.

\newpage

\vspace{0.75cm}

{\Large Generalização}
\begin{figure}[h]
   \centering
   \includegraphics[width=5cm, height=2cm]{Generalization.png}
   \caption{Visual Paradigm, 2023}
\end{figure}

   A generalização indica que um Caso de Uso geral pode ter outras
maneiras específicas de atingir o objetivo que envolve outros Casos de
Uso.

\vspace{0.75cm}

{\Large Incluir (Include)}
\begin{figure}[h]
   \centering
   \includegraphics[width=5cm, height=2cm]{Include.png}
   \caption{Visual Paradigm, 2023}
\end{figure}

   O Include indica que um Caso de Uso depende de outro, por isso
deve inclui-lo no processo para que seu objetivo ou sua ação seja
executada.

\vspace{0.75cm}

{\Large Estender (Extend)}
\begin{figure}[h]
   \centering
   \includegraphics[width=5cm, height=1.5cm]{Extend.png}
   \caption{Visual Paradigm, 2023}
\end{figure}

   O Extend por sua vez tem a intenção de representar um relacionamento
entre Casos de Uso, onde um deles adiciona algum tipo de funcionalidade
extra para o outro.

\newpage

\subsection {Resultados}

Diagrama de Caso de Uso - Sistema Bancário\\
\begin{figure}[h]
   \centering
   \includegraphics[width=16cm, height=19cm]{Diagram.png}
\end{figure}
   
Feito em Visual Paradigm

\newpage

\section {Conclusão}

   A intenção desse Relatório foi colocar em prática o conteúdo Caso
de Uso (UML) e com isso explicar de maneira breve certas funções e
elementos desse modelo da Linguagem Unificada. O trabalho trouxe
também uma visão geral do assunto dando a entender seus pontos fortes e
suas características principais levando em consideração um diagrama
simples de Sistema Bancário.

   Tendo isso em vista o modelo correlaciona simplicidade com
objetividade e traz uma clareza para sistemas básicos de maneira pouco
direta e sem muita dificuldade de entender.

   Conclui-se, portanto, que o Caso de Uso é um ótimo modelo para ser
utilizado em projetos e sistemas mais simples, porém em casos mais
complexos e abstratos talvez usar outro modelo possa ser mais viável.

\newpage

\section {Referências}

\begin{itemize}
   
   \item Astah. Use Case Diagram. \\
Disponível em: astah.net/support/astah-pro/user-guide/usecase-diagram/\\
Acesso em: 29 nov.2023.

   \item Hamilton, Kim/Miles, Russell. Learning UML 2.0.\\ 
O'Reilly: abril, 2006

   \item TRT9. Use Case Model.\\ 
Disponível em: trt9.jus.br. Acesso em 30 nov.2023.

\end{itemize}

\end{document}