\documentclass[12pt, a4paper]{article}
\usepackage{graphicx}
\usepackage{braket}
\usepackage{enumitem}
\usepackage{indentfirst}
\usepackage[margin=1.5cm]{geometry}

\begin{document}

\thispagestyle{empty}

\begin{center}
    
    {\LARGE
        Sistemas de Computação e de Informação - Aula 01\\
    }
    {\Large
        Benay T. da L. de Carvalho - Universidade Norte do Paraná\\
        RA: 3681235802
    }

\end{center}

\vspace*{2cm}

\large
\noindent
\textbf{Regra de Negócio}

Um comportamento de valor específico de um aplicativo que deve ser alcançada
com a execução de um ou mais requisitos* do sistema.

\noindent
\textbf{\\Dado}

É a matéria prima de um sistema. 
São coletados pelos mais diversos tipos de fontes.
Exemplo de dado: números, palavras, sons, nomes, imagens, vídeos, etc.

\noindent
\textbf{\\Informação}

Somente será obtida se dados puderem ser relacionados. 
Informação é o que se obtêm com a aplicação de conhecimento* sobre os dados.
Exemplos de informação: Número de pessoas com mais de 60 anos, 
estudantes com nota maior que 8.

\noindent
\textbf{\\Sistema de Informação ou Sistema Computacional}

Sistemas de Informação são mecanismos que têm a funcionalidade de ajudar o usuário.
Por exemplo, sistema que controla a entrada e saída de estoque de um supermercado 
é um sistema de informação.

\noindent
\textbf{\\Objetivo dos Sistemas de Informação}
\begin{itemize}
    \item Gerar informações específicas.
    \item Manipulação de dados de maneira programada e confiável.
    \item Agilidade na geração de informações.
\end{itemize}

\noindent
\textbf{\\Tecnologia da Informação - TI}

É representada por todos os componentes de hardware e software 
de um sistema computacional.
O papel principal da TI é dar suporte às estratégias 
e aos processos de uma organização. Exemplo: Compras pela internet, 
sistema de estoque de uma loja.

\textbf{\\Requisitos:} São especificações que satisfazem a maneira que o usuário utiliza o sistema.
São as funcionalidades de um sistema.
\textbf{\\Conhecimento:} É a execução do processamento de dados.

\newpage
\vspace*{0.5cm}

\noindent
\textbf{Tecnologias e Estratégias}

\begin{itemize}
    \item \textbf{Disruptiva:} A estratégia disruptiva baseia-se em inovação, mudar completamente
    o sistema ou tecnologia em uso e transformar em algo completamente novo. Um exemplo disso 
    seria uma empresa que utiliza um sistema manual passar a usar um sistema informatizado
    \item \textbf{Sustendada:} Melhora o sistema informatizado atual, focando apenas em trazer 
    funcionalidades novas e/ou aprimoradas
\end{itemize}

\noindent
\textbf{\\Quais Recursos são prioritários para o sucesso?}

\begin{enumerate}[label*=\textbf{\arabic*}.]
    \item \textbf{Pessoas} Primeiro o mais importante é conversar com o cliente/usuário
    saber quais são as expectativas e planos de negócio.
    \item \textbf{Informação} Depois é importante saber quais serão os requisitos, 
    quantas funcionalidades o projeto terá, qual será a dificuldade para fazer, 
    quanto tempo levará, etc.
    \item \textbf{Tecnologia da Informação} A infraestrutura do sistema em si é o que vem por último,
    mas claro, não significa que deverá ser feito de qualquer jeito.
\end{enumerate}

\noindent
\textbf{\\Investigando e conhecendo o sistema}

Nessa fase é feita uma investigação para conhecer o sistema atual que será melhorado
ou planejar o que será feito para criar um sistema do zero.

\noindent
\textbf{\\Métodos Investigativos}

\begin{itemize}
    \item Seminários, questionários, brainstorming.
    \item Observações pessoais e entrevistas com funcionários.
    \item Pesquisas: arquivos, manuais de procedimentos, registros, etc.
    \item Estudo da relação dos costumes pessoais dos funcionários com as atividades e o sistema.
    Como o sistema implementado irá afetar os funcionários de maneira pessoal.
\end{itemize}

\newpage
\vspace*{0.5cm}

\noindent
\textbf{\\Objetivo dos Métodos Investigativos}

\begin{itemize}
    \item Compreender o comportamento do usuário 
    para aprimorar sua experiência com o sistema implementado.
    \item Coletar informações sobre as necessidades para um sistema renovado.
    \item Entender o funcionamento do sistema atual (se existente) para facilitar a 
    mudança para o novo sistema.
\end{itemize}

\noindent
\textbf{\\Quais dados coletar?}

Esse aspecto varia de empresa para empresa, ou usuário para usuário, mas de forma geral 
algumas coisas podem ser esperadas como por exemplo:

\begin{itemize}
    \item Produtos vendidos por dia, mês e ano.
    \item Valores faturados por dia, mês e ano.
    \item Custos, gastos e lucros.
    \item Possíveis melhorias na reposição de produtos.
    \item Códigos referentes aos produtos.
    \item Status de eficiência de estoque mínimo para decisão do setor de compras.
    \item Comportamento econômico para planejar investimentos na loja.
\end{itemize}

\noindent
\textbf{\\Evolução dos Computadores}


\textbf{\\Ábaco}

Cálculo mecânico. Funcionava como uma calculadora.


\textbf{\\Primeira Geração de Computadores(1946-1954)}

Funcionavam a válvula. modelo ENIAC pesava 30 toneladas e ocupava 140m2.


\textbf{\\Segunda Geração(1955-1964)}

Utilizavam linguagem Assembly, em seguida Fortran e depois Pascal.
Armazenamento em disco e fita magnética.

\newpage
\vspace*{0.5cm}

\textbf{\\Terceira geração(1964-1977)}

Uso da linguagem Fortran e Cobol.
Uso de microchips: dezenas de transistores em um único chip.


\textbf{\\Quarta Geração(1977-1991)}

Chip dotado de processamento. Computadores Pessoais.
Sistemas Unix, MS-DOS e Apple Macintosh.


\textbf{\\Quinta Geração(1991-...)}

Processadores de 64 bits e memórias maiores. Discos rígidos de grande capacidade.
Conexão com internet. Multimídia.


\textbf{\\Tecnologias do ano 2000}

\begin{itemize}
    \item Consolidação das redes sociais.
    \item Máquinas Virtuais.
    \item Sistemas cluster*.
    \item Computação em nuvem (cloud computing).
    \item Tecnologias sem fio (Bluetooth por exemplo).
    \item Streaming de multimídia.
\end{itemize}

\noindent
\textbf{\\Recursos Operacionais Estratégicos}
\begin{itemize}

    \item \textbf{Software RIA (Rich Internet Application)} Plataforma de computação em nuvem.
    Pode ser encorporado no sistema informatizado. Exemplos:
    Google Docs, Google Maps, Microsoft Office.

    \item \textbf{Radio Frequency IDentification (RFID)} Identificação por radiofrequência.
    Permite recursos que usam do código de barras, rastreamento de produtos,
    Controle da entrada e saída de estoque.
    \item \textbf{e-Business e e-Commerce}  e-Business é todo processo de negociação online.
    Pode ou não haver comerciação.
    e-Commerce é o processo de transações de compra e venda online.
\end{itemize}

\begin{itemize}
    \item \textbf{m-Business} Negócios eletrônicos móveis, versão móvel do e-Business
\end{itemize}

\newpage
\vspace*{0.5cm}

\noindent
\textbf{\\Modelo de tomada de decisão por George Huber}
\begin{enumerate}[label*=\textbf{\arabic*}.]
    \item \textbf{Inteligência} Potenciais problemas e a limitação dos recursos.
    \item \textbf{Projeto} Soluções alternativas para o problema e suas viabilidades.
    \item \textbf{Escolha} Ações que deverão ser tomadas para alcançar os objetivos.
    \item \textbf{Implantação} Soluções escolhidas são colocadas em prática.
    \item \textbf{Monitoração} Avaliar as soluções implantadas para determinar 
    se os resultados foram como esperados. Fazendo ajustes se necessário.
\end{enumerate} 

\vfill
\begin{center}
    Curitiba\\
    2023
\end{center}


\end{document}