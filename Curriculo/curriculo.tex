\documentclass[12pt]{article}
\usepackage[T1]{fontenc}
\usepackage{lmodern}

% chktex-file 8
% chktex-file 36
% chktex-file 13

\usepackage{enumitem}
\usepackage[a4paper,left=.9in, right=.9in, top=0.9in, bottom=0.5in]{geometry}

% package settings
\usepackage[
    hidelinks,
    pdfnewwindow=true,
    pdfauthor={Benaytomas},
    pdftitle={Curriculum Vitae Benay Tomas},
    pdfpagemode=UseThumbs,
]{hyperref}

\pagestyle{headings}
\markright{\textbf{Benay Tomas da Luz de Carvalho}}

\setlength\parindent{2em}

\thispagestyle{empty}

% define cv section
\newcommand{\cvsection}[1]{\section*{\rmfamily#1}}
\newcommand{\cvsubsection}[1]{\subsection*{\rmfamily\hspace{1.6em}#1}}
\newcommand{\HL}[1]{\textbf{\color{red}#1}}

% begin
\begin{document}

% name
\begin{center}
    \Huge{
    \rmfamily
    \textbf{Benay Tomas da Luz de Carvalho}}
\end{center}
\vspace{5pt}


\setlength{\parskip}{1pt}
\renewcommand{\arraystretch}{1.25}

% Contact Information

\begin{center}

{\large

\noindent (+55) 41 98466-3872

\noindent benaytomas@gmail.com

\noindent Rua Campo Mourão, 180 - Curitiba, Paraná

\noindent 31/08/2004, 19 anos

}

\end{center}

\setlength{\parskip}{3pt}

% Summary / Statement
\cvsection{Resumo}

\indent
{\large

Meu objetivo é construir uma carreira no desenvolvimento de software.\\
\indent Bacharelado de Engenharia de Software.\\
\indent Disponível pela tarde e noite.

}

% Education
\cvsection{Educação}
\indent 
{\large

\textbf{Colégio Estadual São Paulo Apóstolo (CESPA)}

\hspace{2em}Formação Ensino Médio (2022)

\textbf{Universidade Norte do Paraná (UNOPAR)}

\hspace{2em}Graduação Bacharelado em Engenharia de Software.

\hspace{2em}Em andamento.
}

% experiência acadêmica
\vspace*{.15cm}
\cvsection{Competências}
\noindent

{\large

\textbf{LaTeX} \hfill Novembro 2023

\hspace{2em}Capacidade de criar arquivos PDF através de LaTeX

\textbf{Linguagem Python}\hfill Fevereiro 2024

\indent
\hspace{2em}Especializando em Machine Learning e Data Science

\textbf{Cálculo} \hfill Fevereiro 2024

\hspace{2em}Conhecimento em Cálculo Diferencial e Integral

}

\vfill
\begin{center}
    \Large 2024
\end{center}

\newpage
\indent

% interests
\cvsection{Interesses}
\indent

{\large

\textbf{Programação}

\hspace{2em}Python

\hspace{2em}Rust

\hspace{2em}C++

\textbf{Visualização de dados}

\hspace{2em}Pandas

\hspace{2em}Plotly

\hspace{2em}Power BI

\textbf{Machine Learning}

\hspace{2em}Scikit-learn

\hspace{2em}Yellow Brick



}

\indent
\cvsection{Idiomas}
\indent

{\large

Português Brasileiro Nativo.

Inglês Intermediário.

}

\vfill
\begin{center}
    \Large 2024
\end{center}
\end{document}