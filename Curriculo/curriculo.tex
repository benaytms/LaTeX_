\documentclass{article}
\usepackage[utf8]{inputenc}
\usepackage{tgschola}
\usepackage{fancyhdr}


% chktex-file 8
% chktex-file 36
% chktex-file 13

\usepackage{enumitem}
\usepackage[a4paper,left=.9in, right=.9in, top=0.9in, bottom=0.5in]{geometry}

% package settings
\usepackage[
    hidelinks,
    pdfnewwindow=true,
    pdfauthor={Benaytomas},
    pdftitle={Curriculum Vitae Benay Tomas},
    pdfpagemode=UseThumbs,
]{hyperref}

\pagestyle{headings}
\markright{\textbf{Benay Tomas da Luz de Carvalho}}

\setlength\parindent{2em}

\thispagestyle{empty}

% define cv section
\newcommand{\cvsection}[1]{\section*{\rmfamily\textit{#1}}}
\newcommand{\cvsubsection}[1]{\subsection*{\rmfamily\hspace{1.6em}\textit{#1}}}
\newcommand{\HL}[1]{\textbf{\color{red}#1}}

% begin
\begin{document}

\fontfamily{qcs}\selectfont

% name
\begin{center}
    \Huge{
    \rmfamily
    \textbf{Benay Tomas da Luz de Carvalho}}
\end{center}
\vspace{5pt}


\setlength{\parskip}{1pt}
\renewcommand{\arraystretch}{1.25}

% Contact Information

\begin{center}

{\large

\noindent (+55) 41 98466-3872

\noindent \href{https://github.com/benaytms}{github.com/benaytms}

\noindent benaytomas@gmail.com

\noindent Rua Campo Mourão, 180 - Curitiba, Paraná

\noindent 31/08/2004, 20 anos

}

\end{center}

\setlength{\parskip}{3pt}

% Summary / Statement
\cvsection{Resumo}

\indent
{\large

Meu objetivo é construir uma carreira em Análise de Dados.\\
\indent Me especializar em Data Science e Machine Learning.\\
\indent Disponível em qualquer horário.

}

% Education
\cvsection{Educação}
\indent 
{\large

\textbf{\textit{Colégio Estadual São Paulo Apóstolo (CESPA)}} \hfill Formação em 2022

\textbf{\textit{Universidade Norte do Paraná (UNOPAR)}} \hfill Formação em 2026

\hspace{2em}\textit{Engenharia de Software}

\textbf{\textit{Análise de dados com Python (FreeCodeCamp)}} \hfill Completo em 2024

\hspace{2em}\textit{Manipulação de dados, visualização e análise usando Pandas e Matplotlib}
}

% experiência acadêmica
\vspace*{.15cm}
\cvsection{Competências}
\noindent

{\large

\textbf{\textit{Python}}

\hspace{2em}Intermediário

\hspace{2em}\textit{A First Course on Data Structure with Python by Donald R. Sheehy}


\textbf{\textit{Cálculo Diferencial e Integral I}}

\hspace{2em}\href{https://ocw.mit.edu/courses/18-01-single-variable-calculus-fall-2006/}{\textit{Single Variable Calculus (MIT OpenCourseWare)}}


\textbf{\textit{LaTeX}}

\hspace{2em}Avançado

\hspace{2em}\textit{Capacidade de criar arquivos PDF usando LaTeX}


\textbf{\textit{Linux}}

\hspace{2em}Familiaridade avançada com Sistemas \textit{Linux}

\hspace{2em}\textit{Debian, OpenSUSE e Mint}


\textbf{\textit{Pacote Office}}

\hspace{2em}Intermediário

\hspace{2em}\textit{Excel / Word / Calc / Writer}

\textbf{\textit{SQL}}


\hspace{2em}Básico

\hspace{2em}\href{https://www.youtube.com/watch?v=byHcYRpMgI4&t=70s}{\textit{SQLite with Python(FreeCodeCamp)}}

}


\vfill
\begin{center}
    \Large 2024
\end{center}

\newpage
\indent

% interests
\cvsection{Interesses}
\indent

{\large

\textbf{\textit{Linguagens}}

\hspace{2em}Python / R / SQL / Perl / Shell


\textbf{\textit{Análise de dados}}

\hspace{2em}NumPy / Pandas / Matplotlib / .csv

\textbf{\textit{Matemática}}

\hspace{2em}Cálculo / Estatística / Algebra Linear

\textbf{\textit{Machine learning}}

}
\cvsection{Idiomas}
\indent

{\large

\textbf{\textit{Nativo}}

\hspace{2em}Português

\textbf{\textit{Avançado}}

\hspace{2em}Inglês

}

\vfill
    \begin{center}
        {\Large 
        \textit{Curitiba}

        2024
        }
    \end{center}

\end{document}