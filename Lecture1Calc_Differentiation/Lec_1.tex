\documentclass{article}
\usepackage[utf8]{inputenc}
\usepackage[a4paper, total={6.5in, 10in}]{geometry}
\usepackage{graphicx}
\usepackage{amsmath}
\usepackage{setspace}
\usepackage{lmodern}
\fontfamily{lmdh}\selectfont

\title{Lecture 1 Differentiation}
\author{Benay Tomas}

\begin{document}

\maketitle

\indent
\section{What is a Derivative?}

{\Large
\begin{itemize}
    \item Geometric Interpretation
    \item Physical Interpretation
    \item The Importance to all measurements
\end{itemize}
}

\section{How to differentiate any function you want}

\vspace{0.25cm}
\Large
\textbf{Geometric Interpretation}

\vspace{0.5cm}
By the Geometric interpretation, the derivative is
the slope of the tangent line in a point P in any function F$(x$).

But what is a tangent line?

Objective: Find the tangent line to f$(x$) at P$(x_0,y_0$):

\vspace{1cm}
\begin{center}
\includegraphics*[height=5cm, width=8cm]{tangLine.png}

Figure 1: Function with tangent and secant lines
\end{center}

\vspace{1cm}

Picture a secant line that crosses through P and another
random point Q, now imagine this point Q gets closer and closer
to P, the limit of the secant line as the distance
between the two points goes to zero IS the tangent line.

\newpage

Let's consider the distance in the X-axis from P to Q as $\Delta$x.
So P is on $(x_0, f(x_0))$ and Q is on $(x_0+\Delta x, f(x_0+\Delta x))$.

And know to calculate the tangent line we'll use the \textbf{the slope equation:}
m = $(y-y_0$)/$(x-x_0$).

But instead of `y' it will be Q's Y-coordinate, instead of `y0' it'll be
P y-coordinate, `x-x0' will be simply $\Delta$x as it is the X difference already. 

We'll have:

\begin{center}
\begin{math}
    \lim_{\Delta x\to 0} \dfrac{f(x_0+\Delta x) - f(x_0)}{\Delta x} = f'(x_0)
\end{math}
\end{center}

Thus the derivative.

\vspace{1cm}

\Large
\textbf{Example 1. $f(x) = \dfrac{1}{x}$}
\vspace{0.4cm}
\noindent

One thing to keep in mind when calculating the derivative, is to never
plug $\Delta$x = 0, because of the underdetermination of dividing by 0.
So, instead, what we do is manipulate the function and its properties 
to make sure we don't end up dividing something by zero.

\begin{center}
    \includegraphics*[height=3.2cm, width=16.5cm]{example1.png}
\end{center}

{\Large
\textbf{Finding the tangent line}
}

Now that we know the slope of the function is $\dfrac{-1}{x_0^2}$.
We just have to use the equation for the tangent line again and plug in 
the values of the function in x0 and the slope.

And we'll get: 

\vspace{0.4cm}

\begin{center}
\begin{math}
    y = \dfrac{-1}{x_0^2}(x-x_0) + \dfrac{1}{x_0}
\end{math}
\end{center}

Simplifying:

\begin{center}
\begin{math}
        y = -\dfrac{x-2x_0}{x_0^2}
\end{math}
\end{center}

\newpage
\section{Notations}

The derivative can be expressed as $f'(x)$ or $\dfrac{d}{dx}f(x)$ or
$\dfrac{dy}{dx}$ or $\dfrac{df}{dx}$ and D$f$

\section{Properties}

This section will have a separated document file just for it, due to having
an enourmous amount of Calculus Properties, not only derivatives but limits
and integrals too.

\end{document}
