\documentclass{article}
\usepackage[utf8]{inputenc}
\usepackage[a4paper, total={6.5in, 10in}]{geometry}
\usepackage{graphicx}
\usepackage{amsmath}
\usepackage{tgschola}
\usepackage{mathptmx}
\usepackage{fancyhdr}

\newcommand{\minus}{\scalebox{0.8}{$-$}}
\newcommand{\plus}{\scalebox{0.6}{$+$}}
\newcommand{\four}{\scalebox{0.8}{$4$}}

\title{Lecture 2 Limits}
\author{Benay Tomas}

\begin{document}

\fontfamily{qcs}\selectfont

\maketitle

\thispagestyle{empty}

\section{Rate of Change}
\textbf{Average Change:} {\large $\frac{\Delta Y}{\Delta x}$} $-$ Represents the average
change of Y over the change of x.

\noindent
\textbf{Instantaneous Rate:} {\large$\frac{dY}{dx}$} $-$ This, unlike the avg. change,
represents the rate of Y when x is very minimal.

The best way to understand this is with a Physics example;

Let's say we drop a pumpkin from the top of a 80m building.
Now assume the pumpkin position has a formula given by: h = 80 - 5t², where
h is the height of the pumpkin.

We dropped the pumpkin and it took 4s to splash on the ground, with this
information we can gather some pretty interesting data. 

First, the average speed, having passed 80m in 4s, we can safely assume
that the average speed of the pumpkin was of 20m/s. Now let's derivate
the position formula, 80 - 5t², with some simple operations we get
s(t) = -10t as the formula for the speed. So we plug in the 4 seconds and
find that the speed the pumpkin hit the ground with was -40m/s (the negative
sign being because of the perspective we adopted).

This goes to show how important is to know the instantaneous rate of something
and therefore how important the derivative is.

\section{Limits and Continuity}
Firstly we'll start by explaining two types of limits, the easy ones and the
hard ones. These aren't actually official classifications but we'll use it
anyway.

The easy ones we can usually just plug in the value and it's all done.

\textbf{Some examples are:}

{\large
\begin{itemize}
    \item[] $\lim\limits_{x \to 4} \frac{x+3}{x+4} =  \frac{7}{8}$
    \item[] $\lim\limits_{x \to 0} \frac{25x^{2} + 12x + 3}{2x + 4} = 3$
\end{itemize}
}
These limits don't have indeterminations or any kind of complication whatsoever.
They are straight forward, you put the value for which x is approximating towards
and that is it.

Now the hard ones are bit more complicated, as the name suggests. These
limits usually result in some sort of indetermination, for instance:
{\large $\frac{\infty}{\infty}, \frac{0}{0}, 0 \times \infty, 1 \times \infty, \infty - \infty.$}

There's a couple more but these are the most common ones.

To handle these hard ones usually revolves around manipulating the polynomials in a 
way that removes the part that is causing the indetermination.

\textbf{For example:}

{\large
    $\lim\limits_{x \to 3} \frac{x^{2} - 9}{x-3} = \frac{0}{0}$
}

So what we have to do in this case is to expand the numerator, x² - 9;
You may have recalled the 

a² - b² expression, which can be rewritten as (a-b)(a+b)

We'll have: {\large $\lim\limits_{x \to 3} \frac{(x-3)(x+3)}{x-3}$}

Then we cancel out the common $x - 3$ and we'll get:

{\large $\lim\limits_{x \to 3} \frac{(x+3)}{1} $} = 6

\newpage
\subsection{Continuity}
Before introducing the concept of continuity, we'll first see about
left-right-hand limits. 

{\large $\lim\limits_{x \to 0^{\plus}} f(x)$} 

The zero with a plus sign
represents that x is approaching zero from positive numbers, this means
x will be sufficiently close to zero but will always be larger than 
zero and this is called a right-hand limit.

{\large $\lim\limits_{x \to 0^{\minus}} f(x)$} 

Follows the same principle, x will be sufficiently close to zero to be considered
zero but won't actually be zero, it'll be slighty smaller and this is called
a left-hand limit.

\textbf{A function is continuos at C if}

\begin{enumerate}
    \item {\large $\lim\limits_{x \to C} f(x)$} exists
    \item {\large $\lim\limits_{x \to C^{\plus}} f(x)$} =
    {\large $\lim\limits_{x \to C^{\minus}} f(x)$}
    \item {\large $\lim\limits_{x \to C} f(x)$} = $f(C)$
\end{enumerate}

If these 3 happen to be true, then $f(x)$ is continuos at $C$.

\subsection{Discontinuity}
Following the concept of continuity, discontinuity is when in a random
point B the function is not continuos.

There are 4 types of discontinuity. See it below:

\begin{center}
    \includegraphics*[height=6cm, width=16cm]{discontinuities.png}
\end{center}

The Removable discontinuity is when in a function $f(x)$ doesn't exist
the point $f(a)$, the reason for this can vary but usually at that specific
point occurs an indetermination.

The Infinite discontinuity is when the left and right-hand limits coincide
in positive infinity and negative infinity, so the limit at that specific
point doesn't exist, therefore is not continuos.

The Jump discontinuity is the same as the infinite discontinuity, the difference
is that the left and right-hand limits coincide in two different values (that is not infity) 
when x = a. In this case the function can decide in which of the 2 limits
the point $f(a)$ exists, in the picture it the function specified that
the left-hand limit includes the point 'a'.

The Oscillating discontinuity happens usually with trigonometric functions.

\newpage
\thispagestyle{headings}

\section{Differentiabilty implies continuity ($D => C$)}
\underline{\textbf{Definition.}}

\noindent
If a function $f$ is differentiable at
a point C, then the function is continuos at C.

Differentiabilty is the capability of a function to be derived in a 
certain point.

Continuity is when the limit of a function in a certain point exists
and that limit is the same as the function itself in that specific point.

\noindent
\underline{\textbf{Proof.}}

Assume: $f$ is differentiable at C.

{\large $\lim\limits_{x \to C} f(x) - f(C) = \lim\limits_{x \to C} (x-c) \frac{f(x) - f(C)}{x-c}$}

\vspace{0.5cm}

{\large $[\lim\limits_{x \to C} (x-C)] \cdot [\lim\limits_{x \to C} \frac{f(x) - f(C)}{x-c}]$}

\vspace{0.5cm}

{\large $0 \cdot f'(C)$}

\vspace{0.5cm}
So this means that {\large $\lim\limits_{x \to C} f(x) - f(C) = 0$}

\vspace{0.5cm}
With that we can divide the limit as so

{\large $\lim\limits_{x \to C} f(x) - \lim\limits_{x \to C} f(C) = 0$ and $\lim\limits_{x \to C} f(C)$ is simply $f(C)$}

\vspace{0.5cm}
So we get to:

\vspace{0.1cm}

{\large $\lim\limits_{x \to C} f(x) = f(C)$}

\vspace{0.5cm}
This being, if a function $f(x)$ is differentiable at C, then it is continuos
at that same point.

\section*{References}
\begin{enumerate}
    \item Derivatives, limits, sums and integrals in LaTeX - maths.tcd.ie/
    \item Indeterminate forms - calcworkshop.com/limits/
    \item Personal Lecture Notes
\end{enumerate}

\vfill
\begin{center}
    Curitiba

    2024
\end{center}

\end{document}
