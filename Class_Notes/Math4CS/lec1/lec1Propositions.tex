\documentclass{article}
\usepackage[utf8]{inputenc}
\usepackage[a4paper, total={6.5in, 10in}]{geometry}
\usepackage{graphicx}
\usepackage{amsmath}
\usepackage{tgschola}
\usepackage{mathptmx}

\newcommand{\minus}{\scalebox{0.8}{$-$}}
\newcommand{\plus}{\scalebox{0.6}{$+$}}
\newcommand{\four}{\scalebox{0.8}{$4$}}

\title{Lecture 1 Mathematics for Computer Science}
\author{Benay Tomas}

\begin{document}

\fontfamily{qcs}\selectfont

\maketitle


\section{Proof}
Proof is a method for ascertaining the truth.

These are some forms of proving:

\begin{itemize}
    \item Experimentation, Observation
    \item Sampling, Counter Exampling
    \item Common Sense
\end{itemize}

A mathematical proof is a verification of a \underline{propositions}
by a chain of logical deductions from a set of \underline{axioms.}

\subsection{Proposition}

A proposition is a statement
that can be true or false.

\textbf{Examples.}

\begin{itemize}
    \item[] 2+3=5 is True. The sky is
    yellow is False.
    \item[] $\forall n \in N$ [ n² + n + 41 is a prime number ]
    \item[] $\exists a, b, c, d \in N\plus$ [ $a^{4} + b^{4} + c^{4} = d^{4}$ ]
    \item[] 313(x³+y³) = z³ [ has no positive integer solutions ]
    \item[] Goldbach's proposition: Every even positive integer but 2 is the sum of 2 primes.
    \item[] $\forall n \in Z\plus$ [ $n \ge 2 \implies n^{2} \ge 4$ ]
\end{itemize}

The part inside the brackets is called
predicate, and it's a proposition where 
to be truth it depends on a variable value.

And the arrow is called an \underline{implication.}

\subsubsection{Implication}
An implication is a statement
in which one proposition implies another one.\\

For example, let P and Q be two different propositions,

P $\implies$ Q reads as P implies Q.\\

In this case, we have some possibilities for P and Q and for the
statement as whole.
These possibilities are ordely written in what's called a Truth Table:

\begin{displaymath}
    \begin{array}{|c c|c|}
        P & Q & P \implies Q\\
        \hline
        T & T & T\\
        T & F & F\\
        F & T & T\\
        F & F & F\\
    \end{array}
\end{displaymath}

\newpage
\indent
There is also the double implication
or biconditional, in which is exactly 
like the implication but Q also implies P.

The notation is:

$P \iff Q$

And the Truth Table for this one would be:

\begin{displaymath}
    \begin{array}{|c c|c|}
        P & Q & P \iff Q\\
        \hline
        T & T & T\\
        T & F & F\\
        F & T & F\\
        F & F & T\\
    \end{array}
\end{displaymath}

\subsection{Axioms}


Now an axiom is a proposition in which
we assume to be True. We say assume because most of it
is so fundamental and basic that we don't have
ways to prove it, so we assume to be True
by common sense and use it 
to prove others that we actually have to prove.\\

\textbf{Examples.}

If a = b, and b = c, then a = c.

If A $*$ B = 0, then A=0 or B=0 or A=B=0.\\

A general rule about axioms is that they must
be consistent and non-redundant.

Being consistent means it should be impossible for an axiom
to be True and False at the same time.
And non-redundant means they must mean something and must be useful to proof.

\end{document}