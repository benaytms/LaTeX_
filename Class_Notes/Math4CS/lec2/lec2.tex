\documentclass{article}
\usepackage[utf8]{inputenc}
\usepackage[a4paper, total={6.5in, 10in}]{geometry}
\usepackage{graphicx}
\usepackage{amsmath}
\usepackage{tgschola}
\usepackage{mathptmx}
\usepackage{fancyhdr}

\newcommand{\minus}{\scalebox{0.8}{$-$}}
\newcommand{\plus}{\scalebox{0.6}{$+$}}
\newcommand{\four}{\scalebox{0.8}{$4$}}

\title{Lecture 2 Mathematics for Computer Science}
\author{Benay Tomas}

\begin{document}

\fontfamily{qcs}\selectfont

\maketitle

\thispagestyle{empty}

\section*{Continuation of Lecture 1}

\section{Proof by Contradiction}
\hrule
\vspace{0.2cm}
Essentially:
To prove $P$ is true, we assume $P$ is false
(i.e., $\neg P$ is True) then use that 
hypothesis to derive a contradiction;
That is, if $\neg P \implies F$ ; then
$P$ must be $T$.

\subsection*{\underline{Example}}

Prove that $\sqrt{2}$ is irrational.
\vspace{0.2cm}
\hrule
\subsubsection*{\underline{Proof by Contradiction}}

Assume for purpose of contradiction
that $\sqrt{2}$ is rational.

Then this means, by the rational number
definition, that 2 must be able to
be represented by the division of 2
values in their lowest terms. So:

\begin{center}
  $\implies \sqrt{2} = \frac{A}{B}$ \\
  \vspace{0.2cm}
  $\implies 2 = \frac{A^{2}}{B^{2}}$ \\
  \vspace{0.2cm}
  $\implies 2B^{2} = A^{2}$ \\
  \vspace{0.2cm}
  $\implies$ This means 'A' is even \\
  \vspace{0.2cm}
  $\implies$ Which means it can divide 4\\
  \vspace{0.2cm}
  $\implies$ So $\frac{4}{A^{2}} = \frac{4}{2B^{2}}$ \\
  \vspace{0.2cm}
  $\implies$ Therefore $2B^{2}$ implies that B is also even \\
  \vspace{0.2cm}
  $\implies$ So $\sqrt{2}$ can't be $\frac{A}{B}$ because there's no
  even value minor than $2$. \\
  \vspace{0.2cm}
  $\implies$ Thus a contradiction.
\end{center}

\subsubsection*{\underline{Conclusion}}

$\sqrt{2}$ is irrational.

\section{Proof by Induction}
\hrule
\vspace{0.2cm}
Let $P(n)$ be a predicate. If $P(0)$
is true and $\forall n \in N$ |
$(P(n) \implies P(n+1))$ is true then
$\forall n \in N$ | $P(n)$ is true, if
$P(0) \implies P(1), ...,$ are true;
then $P(0), P(1), P(2), ..., P(n+1)$ are true.

\vspace{0.2cm}
\noindent
\subsection*{Examples}
\hrule
\vspace{0.2cm}

In this section i'll express 2 examples
of use for the Induction method.
The first one being the Gauss Sum and the second
a special arithmetic expression.

\newpage
\subsection{Gauss Sum}
\vspace{0.2cm}
\hrule
\vspace{0.2cm}
The Gauss Sum states that \[
\sum_{i=0}^{n} i = \frac{n(n+1)}{2}\]

\vspace{0.2cm}
\subsubsection*{\underline{Proof by Induction}}
\vspace{0.2cm}

Let $P(n)$ be the proposition that 
encapsulates this expression. 
To prove it we'll need to do 2 steps
the base case and the inductive step.

\subsubsection*{Base Case}
\hrule
\vspace{0.2cm}
The predicate is tested at the lowest
value for $n$ possible

\[\sum_{i=0}^{0} i = \frac{0(0+1)}{2} = 0\]

So it is true.

Then we now go for the Inductive Step.

\subsubsection*{Inductive Step}
\hrule
\vspace{0.2cm}
For $n \geq 0$, show $P(n) \implies
P(n+1)$ is True.

To do this we'll assume $P(n)$ is true for purposes of
induction. 

Therefore, assume \[\sum_{i=0}^{n} i = \frac{n(n+1)}{2}\]

With this we need to show that

\[\sum_{i=0}^{n+1} i = \frac{(n+1)(n+2)}{2}
\implies \frac{n(n+1)}{2} + n+1\]

If we expand the expression on the right
we'll get
\[\frac{n^{2} + n + 2n + 2}{2}\]

Which is none other than

\[\frac{n+1(n+2)}{2}\]

\subsubsection*{Conclusion}
\hrule
\vspace{0.2cm}
This shows that using $P(n)$
we can inductively prove that the
expression is valid for $P(n+1)$,
and so therefore it is valid for any $n$.

\newpage
\noindent
\subsection{Special Arithmetic Expression}
\hrule
\vspace{0.2cm}
Consider the following:

\[\forall n \in N : 3|(n^{3}-n)\]

\begin{center}
Where the value inside the parenthesis
means it's divisible by 3.
\end{center}

\vspace{0.2cm}
\subsubsection*{\underline{Proof by Induction}}
\vspace{0.2cm}

$Let P(n)$ be $3|(n^{2} - n)$

\subsection*{Base Case}
\hrule
\vspace{0.2cm}
The value for $n$ can be 0 for this case.
\[3 | (0^{3} - 0) = 0\]
\[0/3 = 0 \implies True\]

\subsection*{Inductive Step}
\hrule
\vspace{0.2cm}

For $n > 0$, show that $P(n) \implies
P(n+1)$ is True.

Assuming $P(n)$ is True, for $n+1$ we would have
\[(n+1)^{3} - (n+1)\]
\[\implies n^{3} + 3n^{2} + 3n + 1 - n - 1\]
\[\implies n^{3} +3n^{2} + 2n\]
\[\implies n^{3} - n + 3n^{2} + 3n\]

\subsection*{Conclusion}
\hrule
\vspace{0.2cm}
Considering $n^{3} - n$ and $3n^{2} + 3n$ are both divisible by 3,
then there's our answer, the proposition holds.

\newpage
\thispagestyle{headings}
\section{References}
\begin{enumerate}
  \item My personal notes taken
  \item Slides Lecture 2 Mathematics for Computer Science - MITOpenCourseware
  \item Overleaf LaTeX Help
\end{enumerate}

\vfill
\begin{center}
  Curitiba

  2024
\end{center}

\end{document}
