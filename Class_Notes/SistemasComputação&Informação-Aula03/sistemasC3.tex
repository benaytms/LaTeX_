\documentclass[12pt]{article}
\usepackage{graphicx}
\usepackage{braket}
\usepackage{enumitem}
\usepackage{lmodern}
\usepackage[T1]{fontenc}
\usepackage[margin=1.5cm]{geometry}
\usepackage{textcomp}
\usepackage{sectsty}

\usepackage[
    hidelinks,
    pdfnewwindow=true,
    pdfauthor={Benaytomas},
    pdftitle={Benay Tomas Class Notes},
    pdfpagemode=UseThumbs,
]{hyperref}

\sectionfont{\fontsize{13}{15}\selectfont}
\setlength{\parindent}{1cm}

\begin{document}

\begin{center}
    {\large
    \textbf{Sistemas de computação e informação - Evolução dos Sistemas
    Online\\}
    Benay T. da L. de Carvalho - Universidade Norte do Paraná\\
    28 de fevereiro de 2024
    }
\end{center}

\indent
\section{O que será abordado?}
A evolução dos sistemas de informação por meio do uso da internet.
As novas tecnologias que estão mudando a forma
de utilização de aplicativos que são executados
diretamente por meio da internet. Os sistemas
de colaboração e ferramentas de social business, que
estão mudando a forma de trabalho dos usuários
com uso de sistemas que permitem o trabalho de equipes
de forma colaborativa.

\indent
\section{Evolução dos sistemas de informação com a internet}
A internet foi criada no início da década de 1970 como uma rede do Departamento
de Defesa dos Estados Unidos com intenção de conectar cientistas e professores
universitários ao redor do mundo. Sendo o maior caso de implementação de 
redes individuais. 

Uma vez que a informação está disponível a todos,
a internet aumenta o poder de barganha dos clientes, que podem com grande rapidez
e facilidade encontrar um fornecedor com custos e/ou condições de fornecimento
muito mais atrativas, fazendo que os lucros das empresas sejam reduzidos e impondo
severos riscos a alguns setores da economia, por exemplo as enciclopédias impressas
e as agências de turismo.

Mas a internet também possibilitou a criação de inúmeros novos serviços e mercados,
constituindo-se atualmente como alicerce de muitos modelos de novos negócios,
fazendo que setores inteiros da economia mudem sua maneira de fazer negócios em
escala global. Veremos agora alguns desses serviços e tecnologias.

\indent
\subsection{Computação em nuvem}
Também conhecida como cloud computing, é uma tecnologia que permite o uso da internet
para armazenamento de dados (textos, imagens, vídeos, jogos, filmes, etc.) e
execução de alguns softwares sem que o usuário necessite instalar programas em seu
computador.

Um sistema de computação em nuvem pode ser: \textbf{Público} quando o serviço 
é mantido por um provedor de serviço em nuvem. \textbf{Privado} quando o serviço 
é mantido exclusivamente por uma organização.

Os ambientes de Computação em nuvem estão sempre sujeitos à atuação de três
importantes forças que, juntas, facilitam ou dificultam a adesão dos usuários. Essas
são: \textbf{Barreiras potenciais} à adoção da computação em nuvem. 
\textbf{Benefícios potencias} da adoção da computação em nuvem. 
\textbf{Riscos inerentes} à computação em nuvem.

\subsection{Ferramentas e tecnologias para colaboração e Social Business}
Uma cultura colaborativa orientada a equipes não gera benefício se não
existirem sistemas que viabilizam a colaboração e o Social Business.

Atualmente, existem algumas ferramentas avançadas como Lotus Notes
porém são pagas. Outras estão disponíveis gratuitamente e são adequadas
para empresas pequenas.

Exemplo dessas ferramentas são os Wikis, que nada mais são do que ferramentas
que permitem adicionar, alterar e remover um conteúdo sem qualquer conhecimento
sobre desenvolvimento de páginas web ou técnicas de programação. 
São conteúdos armazenados e organizados em um ambiente online. Maior exemplo
é a Wikipédia, maior projeto de edição colaborativa do mundo.

Outros exemplos são: Emails, Whatsapp, Sharepoint, SmartCloud for Business, 
Discord, Dropbox, Google Drive, iCloud, FaceTime, etc.

\indent
\section{Tecnologia de desenvolvimento de sistemas}
Quando pensamos na arquitetura de um sistema de informação, devemos primeiro
entender quais são as necessidades dos usuários e a que o sistema se presta, ou seja,
quais os propósitos do sistema, antes de começarmos a projetar o sistema em si.
Você pode observar que um sistema de informação nada mais é que uma ferramenta
a serviço de um instituição.

\subsection{Fundamentos gerais sobre desenvolvimento de sistemas}
\textbf{Arquitetura de um sistema de informação é composta por 3 aspectos básicos}
\begin{itemize}
    \item \textbf{Estrututura do sistema:} Framework, que são as funcionalidades básicas
    do sistema.
    \item \textbf{Componentes:} Os componentes são os módulos do sistema. Os
    módulos são organizados pelas funcionalidades que agrupam.
    \item \textbf{Relacionamentos:} O relacionamento entre os componentes, isto é,
    as comunicações, trocas de dados e informações entre os componentes.
\end{itemize}

Podemos ainda considerar que a arquitetura de sistemas de informação é um conjunto
de várias arquiteturas que se completam: Arquitetura de hardware, de software, corporativa,
de sistemas colaborativos, de sistemas de manufatura, de sistemas estratégicos.

O site TutorsGlobe estabelece cinco questões para definição da arquitetura de um
sistema de informação. Essas perguntas devem ser expandidas dependendo do caso trabalhado.

\begin{enumerate}[label=\textbf{\arabic*}.]
    \item Quem serão os usuários do sistema?
    \item Para que servirá o sistema e quais serão seus principais processos?
    \item Como serão as redes que transportarão os dados e informações para dentro
    e fora do sistema?
    \item Que eventos e em que momento do tempo esses eventos deverão ocorrer 
    no sistema?
    \item Quais são as regras e casos de uso do sistema?
\end{enumerate}

\indent
\section{Sistemas de informação hospedados na nuvem}
Esses sistemas possuem algumas particularidades importantes acerca de sua arquitetura:
\textbf{O backend} fica hospedado não mais em servidores pertencentes à instituição, 
mas em datacenters. Permanecem privados, mas a infraestrutura não é de propriedade 
da instituição dona dos sistemas de informação. 

\textbf{O acesso aos sistemas} não feito 
por meio de linhas privativas de comunicação, contratadas diretamente dos provedores 
de comunicação (operadoras) do país. 
A comunicação é feita por meio de canais públicos de comunicação (internet).

\textbf{O frontend} dos sistemas passam a ser websites acessíveis por meio de qualquer navegador
ou por meio de aplicativos móveis gratuitos.

\indent
\section{Classificações e tipos de linguagens de programação}
Veremos sobre Linguagem de programação, linguagem de máquina, 
categorização, paradigma de programação.

A linguagem de máquina é composta por cadeias de números expressos em base binária
e age diretamente sobre o processador. Os comandos possíveis em linguagem de máquina
são bastante simples e são necesários muitos deles para realizar alguma ação 
que tenha significado para nós, humanos.

Alguns exemplos de comandos possíveis em linguagem de máquina são:
\textbf{Buscar e escrever} dados em alguma posição de memória.
\textbf{Realizar operação lógica.}
\textbf{Comparar dois valores} e informar qual é maior ou menor.
\textbf{Enviar dado} para manipulação de algum periférico.

Porém, sendo muito complexa e demorada de usar para nós humanos, a linguagem
de programação serve para traduzir o que queremos para linguagem de máquina.

\subsection{Categorização}
A categorização de linguagens de programação é um processo arbitrário: Podemos
categorizar de inúmeras formas. Uma maneira é categorizar pela maneira que a linguagem
se relaciona com o processador. 

Linguagens que geram código que acessa diretamente o 
processador são chamadas de linguagens compiladas, enquanto linguagens que precisam
de programas intermediários que transformem suas instruções em comandos compreensíveis
para o processador são linguagens interpretadas.

O Hyper Text Markup Language, ou HTML, é um exemplo de linguagem interpretada. O C++
utilizado para criar programas executáveis, é uma linguagem compilada.

Dentre as várias maneiras de categorizar uma linguagem, podemos citar:

\textbf{Categorização histórica, categorização por paradigmas e categorização por
acesso direto ou indireto ao processador}

\subsection{Paradigmas de programação}
\begin{itemize}
    \item \textbf{Paradigma de linguages imperativas\\}
    Foco da execução e solução está em como deve ser feito.
    Linguagens: Pascal, Fortran e Cobol.
    \begin{itemize}
        \item \textbf{P. de l. procedural\\}
        Organizado como procedimentos que o programador cria que serve como 
        passo-a-passo que a máquina deve cumprir.
        Linguagens: Perl, Php, Lua.
        \item \textbf{P. de l. orientadas a objetos\\}
        Organizado como objetos que possuem classes e comportamentos associados.
        Principais linguagens: Java, Python e C++.
    \end{itemize}

\newpage

    \item \textbf{Paradigma de linguagens declarativas\\}
    Este paradigma está mais interessado em "o que" e menos no "como"
    Declara verdades lógicas imutáveis.
    Linguagens: HTML, XML, XAML.
    \begin{itemize}
        \item \textbf{P. de l. lógica\\}
        Processo que chega aos resultados através de análise lógico-matemáticas.
        Principais elementos: Proposições, regras de inferência e busca.
        Linguagem: Prolog.
        \item \textbf{P. de l. funcional\\}
        Destaca o uso das funções e na hora de resolver um problema, divide em blocos
        onde são implementadas as funções.
        Linguagens: Lisp, Scheme e Haskell.
    \end{itemize}
\end{itemize}

\noindent
Além do que vimos, \textbf{devemos:}

\begin{itemize}
    \item Saber quais informações devem fazer parte do banco de dados. 
    \item Definir como fazer a modelagem dos dados, de forma a obter um 
conjunto de relacionamentos entre eles, para que o banco de dados tenha uma 
lógica coesa.

    \item Decidir qual bando de dados mais adequado para o sistema.

    \item Identificar o melhor local e o melhor modo de acesso aos dados armazenados no banco
(e se esse banco deve ser centralizado ou distribuído).

\end{itemize}

\indent
\section{Principais tipos de sistemas gerenciadores de banco de dados}
Gerenciamento de dados e informações envolve: 

Determinar quais informações serão necessárias,
adquirir as informações necessárias, organizar as informações, garantir a qualidade da informação e 
disponibilizar ferramentas de acesso aos colaboradores.

\textbf{Hierarquia de dados}

Se trata de uma forma de organizar os dados desde seus elementos constituintes mais simples
até os mais complexos. Os elementos da Hierarquia de dados são:
\textbf{Caractere, Campo, Registro, Arquivo e Banco de dados.}

\textbf{Caractere} são letras e números.
\textbf{Campos} são combinações de letras, números, símbolos.
\textbf{Registros} são conjuntos de campos relacionados que descrevem uma entidade.
\textbf{Arquivo} é um conjunto de registros relacionados.
\textbf{Banco de dados} é uma coleção de arquivos integrados e relacionados.

\indent
\section{Modelagem de dados}
A modelagem de dados é parte integrante da organização da estrutura lógica dos dados.
Consiste em 4 passos, ao final dos quais o modelo de dados terá sido criado e poderá
ser implantado em um SGBD.

\begin{enumerate}[label=\textbf{\arabic*}.]
    \item Identificar os dados que devem fazer parte do banco de dados.
    \item Identificar as relações quantitativas entre os dados(1 para n, n para 1, n para n)
    \item Identificar as relações qualitativas entre os dados
    \item Refinar a representação dos dados, identificando os níveis de abstração e também ajustando os relacionamentos.
\end{enumerate}

\indent
\subsection{Modelo Entidade Relacionamento (MER)}
Definidos os dados e suas relações, o modelo é feito por um diagrama visual.
A figura a seguir apresenta um exemplo desse modelo:

\hspace{5cm}\includegraphics[scale=0.5]{MER.png}

Os modelos de banco de dados relacional mais populares armazenam as informações
em tabelas bidimensionais, chamadas \textbf{relações.} Os principais sistemas de banco 
de dados são: 

\textbf{Access, MySQL, DB2, Oracle e AWS}

\indent
\section{Ferramentas CASE}
São ferramentas computadorizadas que automatizam processos e integram resultados
e subsídios das várias fases do projeto de desenvolvimento de um sistema de informação.

\textbf{Os objetivos das CASE são:}

Auxiliar no desenho do sistema a ser desenvolvido. Auxiliar na definição, acompanhamento
e controle das tarefas de desenvolvimento do sistema. Reduzir a complexidade do projeto
por meio de sua divisão em porções menores e mais facilmente gerenciáveis.
Facilitar a manutenção futura do sistema. Aumentar a consistência do projeto e a coesão
dos módulos do sistema por meio de ferramentas de verificação.

\subsection{Categorização das CASE}
Há dois tipos de categorização. O primeiro é a categorização em termos da fase do projeto
em que as ferramentas são usadas. 

Nesse caso são três categorias:

\textbf{Ferramentas Upper-CASE, ferramentas Lower-CASE e ferramentas Integrated-CASE}

\section*{Tutorial de ferramentas CASE:}

\url{https://www.youtube.com/watch?v=kKsdVFOcr0s}

\end{document}