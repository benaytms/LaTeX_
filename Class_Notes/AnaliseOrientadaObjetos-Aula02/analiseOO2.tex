\documentclass[12pt, a4paper]{article}
\usepackage{graphicx}
\usepackage{braket}
\usepackage{enumitem}
\usepackage{indentfirst}
\usepackage[T1]{fontenc}
\usepackage[margin=1.5cm]{geometry}

\begin{document}

\begin{center}

    {\LARGE
        Análise Orientada a Objetos - Aula 02\\
    }
    {\Large
        Benay T. da L. de Carvalho - Universidade Norte do Paraná\\
        RA: 3681235802
    }

\end{center}

\vspace{1cm}

\large

\noindent
\textbf{Diagramas de casos de Uso, classes e atividades.} 

\begin{itemize}
    \item Diagrama de Casos de uso: é um diagrama comportamental com alta abstração, usado em requisitos.
    \item Diagrama de Classes: são os mais usados da UML, mostra a relação entre classes.
    \item Diagrama de Atividades: mostra o fluxo das ações e atividades de um sistema, 
    é uma variação do caso de uso.
\end{itemize}

\indent
\textbf{\\Diagrama de Casos de Uso}

\begin{enumerate}[label*=\textbf{\arabic*}.]

    \item Elipse (caso de uso): verbo no infinitivo + substantivo
    \item Ator (Objetos/Entidades): apenas o nome
    \item Conexões: relações entre casos de uso
    \begin{enumerate}[label*=\textbf{\arabic*}.]
        \item Associação: associa diretamente um ator com caso de uso
        \item Generalização: herda elementos de um caso para outro(s)
        \item Extensão/Incluir: incluir é obrigatoriedade, extensão é opcional
    \end{enumerate}
    \item Multiplicidade: quantidade de atores irão utilizar um caso de uso e vice-versa
\end{enumerate}

\indent
\textbf{\\Diagrama de Classes}

\begin{enumerate}[label*=\textbf{\arabic*}.]

    \item Objeto: é a instância da Classe
    \item Atributos: características do Objeto
    \item Métodos: funcionalidades do Objeto
    \item Encapsulamento
    \begin{enumerate}[label*=\textbf{\arabic*}.]
        \item Publico(+) Public: todos têm acesso aos atributos
        \item Privado(-) Private: apenas a classe do objeto pode mexer nos atributos
        \item Protegido(\#) Protected: funciona igualmente ao Privado 
        com adicional que as classes filhas também têm acesso aos atributos
        \item Geralmente os atributos são privados e os métodos públicos mas isso pode variar
    \end{enumerate}

\end{enumerate}

\newpage

\indent
\textbf{\\Tipos de Classe}

\textbf{\\Abstrata}
\begin{itemize}
    \item Representada pelo nome em itálico
    \item Não geram instâncias, servem apenas para representar hierarquia e generalização
\end{itemize}

\textbf{\\Interface}
\begin{enumerate}[label*=\textbf{\arabic*}.]
    \item Representada com <<nomeInterface>>
    \item Representa um conjunto de elementos públicos 
    que devem ser implementados por outras classes
    \item Relacionamentos
    \begin{itemize}
        \item Herança: herda funcionalidades da classe pai para classe filha. Representada
        por uma seta cheia
        \item Implementação: as classes filhas irão implementar a classe interface pai,
        vão instanciar a classe pai, de certa forma. Representada por uma seta tracejada
    \end{itemize}
\end{enumerate}

\indent
\textbf{\\Multiplicidade}
Multiplicidade representa os limites inferior e superior da quantidade
de objetos aos quais outro objeto está associado. Esses valores podem variar de
0 à * (que representa muitos) 

\begin{itemize}

    \item 1 - Exatamente um
    \item 1...* - No mínimo 1, no máximo vários
    \item 0...1 - No mínimo 0, no máximo 1

\end{itemize}

\indent
\textbf{\\Composição e Agregação}

\begin{itemize}

    \item Agregação: agregar um objeto a outro, de maneira não obrigatória.\\
    Similiar ao Extender do Caso de Uso. Representado por uma seta de losango.
    \item Composição: exigi um objeto para outro existir, de maneira obrigatória.\\
    Similar ao Incluir do Caso de Uso. Representado por uma seta de losango cheia.

\end{itemize}

\newpage

\indent
\textbf{\\Estereótipos}

\begin{enumerate}[label*=\textbf{\arabic*}.]

    \item <<boundary>>: representa a interface do sistema 
    \item <<control>>: controle da aplicação, 
    serve de intermediário entre as classes boundary e entity
    \item <<entity>>: contém informações geradas ou recebidas pelo sistema

\end{enumerate}

\indent
\textbf{\\Diagrama de Atividades}
Esses são os elementos mais comuns no Diagrama de Atividades:

\begin{enumerate}[label=\textbf{\arabic*}.]

    \item \textbf{Inicial:} Representa o início do diagrama e é obrigatório(circulo fechado)
    \item \textbf{Final:} Representa o fim de um fluxo, é usado para o finalizar o diagrama, 
    pode ser usado mais de uma vez em diferentes circunstâncias
    (circulo fechado com contorno)
    \item \textbf{Terminal:} Representa o fim de um fluxo, porém não encerra o diagrama,
    apenas aquele fluxo específico (circulo com X no meio)
    \item \textbf{Decisão:} Elemento que indica uma condicional e algumas vezes causa um loop
    (losango)
    \item \textbf{Transição:} Representa uma transição de uma atividade para outro elemento
    (seta)
    \item \textbf{Fork/Join:} Fork divide uma transição em múltiplas transições. Join junta
    múltiplas transições em uma só, possuem efeitos opostos basicamente. (ambos são representados com uma linha dividindo o antes e depois das transições)
    \item \textbf{Swimlanes:} Divisões colocadas no diagrama para organizar o momento
    das atividades. 

\end{enumerate}

\indent
\textbf{\\Referências}

\begin{enumerate}[label=\textbf{\arabic*}.]
    
    \item Anotações da aula 2 de análise orientada a objetos
    \item https://pt.wikipedia.org/wiki/UML
    \item https://spaceprogrammer.com/uml/

\end{enumerate}
\end{document}