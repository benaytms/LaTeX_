\documentclass{article}
\usepackage[utf8]{inputenc}
\usepackage[a4paper, total={6.5in, 10in}]{geometry}
\usepackage{graphicx}
\usepackage{amsmath}
\usepackage{tgschola}
\usepackage{mathptmx}
\usepackage{fancyhdr}

\newcommand{\minus}{\scalebox{0.8}{$-$}}
\newcommand{\plus}{\scalebox{0.6}{$+$}}
\newcommand{\four}{\scalebox{0.8}{$4$}}

\title{Lecture 2 Introduction to Computer Science}
\author{Benay Tomas}

\begin{document}

\fontfamily{qcs}\selectfont

\maketitle

\thispagestyle{empty}

\textbf{Subjects Seen}

String object type, branching and conditionals,
indentation and iteration/loops

\section{Strings}
Letters; Special characters; Spaces; Digits

Strings variables need to be enclosed in quotation marks, or single quotes.

\textbf{greetings} = "hello there"

\noindent
To concatenate strings we use "+".
For example,

name = "Ana"

phrase = greetings + " " + name

\noindent
This will give us

phrase = hello there Ana

\subsection{Input and Output}
To output information to the console
we use the \textbf{'print'} built-in function.

\textbf{Example.}

x = "this is a written message"

print(x)

Output: this is a written message

\noindent
To receive information instead of outputting
we use the \textbf{input} built-in function.

\textbf{Example.}

text = input("Type anything: ")
user: anything

print(text)

Output: anything

\section{Comparison Operators}
Consider i and j are variables names of int or string type.

\noindent
Float could be used too but is not advised, as they are volatile when trying to compare it.

See below some comparison examples.

i $>$ j or i $<$ j

i $>=$ j or i $<=$ j

i $==$ j

i $!=$ j

They are evaluated and returns \textbf{True} or \textbf{False.}

\section{Logic Operators}
Let a and b are variable names (with Boolean values)

\begin{itemize}
    \item and: a and b, means a and b need to be True to be True.
    \item or: a or b, means at least one of them needs to be True to the expression to be True.
    \item not: not a, not b, means whatever a or b is it will be reversed.
\end{itemize}

\newpage
\pagestyle{headings}

\section{Branching}
To control flow we need Branching, and for this
we use the \textbf{if-else} expression.

\textbf{if-else} structure:

if $<condition>$:\\
    \hspace*{0.85cm}$<expression>$

else:\\
    \hspace*{0.85cm}$<expression>$

\noindent
If the condition is not matched, it'll execute
the expression inside the else.

We may also have the \textbf{elif} statement,
this will be basically an \textbf{else} statement but 
with a condition to be met, so we can have multiple
branching.

if $<condition>$:\\
    \hspace*{0.85cm}$<expression>$

elif $<condition>$:\\
    \hspace*{0.85cm}$<expression>$

else:\\
    \hspace*{0.85cm}$<expression>$


\section{Indentation}
Indentation is the base of Python.

Differently from other languages, Python doesn't use
curly brackets to indent statements, it use tabs
indentation.

\section{While-loops}
The while structure works like an if-else except it keeps
repeating the expresisons inside the while statement until
the condition is not met (False).

\textbf{While Structure}

while $<condition>$:\\
    \hspace*{1.5cm}$<expression>$\\
    \hspace*{1.5cm}...

\section{For-loops}
The for-loop structure works as a form of iteration
with a known number of times you want the loop to last.

\textbf{For Structure}

for $<iterable-variable>$ in $<range>$:\\
    \hspace*{1.05cm}$<expression>$

This structure can be highly changed, you can have more than
one variable iterating, if having the accordingly range;
You can have the range to be with the built-in function range,
where you may specify the start, stop and increment. But 
you can also have another methods of range, this will depend
on your needs.

\textbf{Break Statement}

The break statement is a built-in function that'll exit 
the current loop you might be in.

\vfill

\begin{center}
    Curitiba

    2024
\end{center}



\end{document}