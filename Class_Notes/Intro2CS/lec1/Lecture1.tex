\documentclass{article}
\usepackage[utf8]{inputenc}
\usepackage[a4paper, total={6.5in, 10in}]{geometry}
\usepackage{graphicx}

\title{Lecture 1 Computation in Python}
\author{Benay Tomas}

\begin{document}

\maketitle

{
    
\Large

\section*{What does a computer do?}
\subsection*{Fundamentally:}
Performs calculations, a billion per second!

\subsection*{What kinds of calculations?}

\begin{itemize}
    \item Built-in of most languages like subtraction, addition, 
    multiplication and division.
    \item Ones you define as the programmer.
    \item Others you might import made by others.
\end{itemize}

\subsection*{Types of Knowledge}

\begin{itemize}
    \item Declarative Knowledge: statements of facts.\\
    Ex: Simon did that.\\
    In programming this would be the variables.
    \item Imperative Knowledge: command, instructions.\\
    Ex: Simon, do that.\\
    In programming this would be the expressions and 
    calculations that generate variables and results.
\end{itemize}

\subsection*{What is a recipe?}

A sequence of steps with a flow control process that
specifies when each step is executed and has a mean 
of determining when to stop.

Recipes are also called algorithms.

\subsection*{Stored Program Computer}

Sequence of instructions stored inside computers.
Built from predefined set of primitive instructions.

1. Arithmetic and logic. 
2. Simples tests. 
3. Moving data. 

\newpage

Special programs executes each instruction in the order
you planned, if you declare a variable after using it,
the computer won't understand, so you need to do everything
you need on a specific order.

\subsection*{Aspects of Languages}

\begin{itemize}
    \item Primitive Constructs: The most basic aspect 
    of languages, in English it's the words. In programming
    it's numbers, strings and operators.
    \item Syntax: Another aspect of languages is the Syntax,
    this will define the difference every language. If you
    use the incorrect syntax the interpreter won't be able
    to understand what you're trying to do.
    \item Semantics: Is which syntatically expressions
    also have meanings. For instance you can do the expression:\\
    3 + `a' and this is syntatically correct because there's
    2 objects and 1 operator between them, but it doesn't make sense.\\
    3 + 5 is syntatically correct and have semantics sense.
\end{itemize}

\subsection*{Python programs}

A program is a sequence of definitions and commands.
Commands are executed by the Python interpreter.
To make programs you can use IDE's there you may use
a Shell or the Editor. Shell is used for more basics
and short programs, the Editor is used for the opposite.

\subsection*{Objects}

Python is a programming language object oriented.
This means everything is an object from a class, 
and every class has a series of methods and atributes.

\begin{itemize}
    \item Programs can manipulate data objects.
    \item Objects have a type that defines the kinds
    of things programs can do to them.\\
    Integers can make mathematical operations.\\
    Lists can have multiple single values.
    \item Objects can be:\\
    Scalar and non-Scalar.
    Scalar objects are objects that are associated with just one value.
    While non-scalar objects can be associated with multiple values, like Lists.
\end{itemize}

\subsection*{Scalar Objects}

\begin{itemize}
    \item int: represent integers, ex: 5.
    \item float: represent real numbers, ex: 3.42.
    \item bool: represent boolean values True and False.
    \item NoneType: represent no type, None.
    \item $type()$: shows the type of an object.
\end{itemize}

Its possible to convert objects of one type
to another.

$float(3)$ = converts int 3 to float 3.0.

$int(3.9)$ = truncates float 3.9 to int 3.

\vspace*{0.30cm}

to show an output from code to user, there's the 
$print()$ command:

$print(3+5)$

output: 8

\subsection*{Expressions}

\begin{itemize}
    \item Combine objects and operators to form it.
    \item An expression has a resultant-value, which
    has a type that depends on its objects and in case
    of the division it will always turn into float.
    \item Syntax for a simple expression: $<object> 
    <operator> <object>$
\end{itemize}

\subsection{Operators}
\
\begin{itemize}
    \item i + j = sums i and j.
    \item i $-$ j = subtracts j from i.
    \item i * j = multiplies i with j.
    \item i / j = divides i with j.
    \item i // j = rounds the division with the nearest whole number.
    \item i \% j = returns the remainder of i/j.
    \item i ** j = i to the power of j.
    \item There's also the logic operators but we'll see them later.
\end{itemize}

\newpage

\subsection*{Binding variables and values}

We use the equal sign `=' to represent assignment of a value to a variable.
In this format: $ <variablename> $ = $ <value> $

For example: $pi = 3.14159$

The value $3.14159$ will be associated with the variable $pi$ and it will be
store in computer memory. By typing $pi$ we will be able to use the value
whenever needed in the program. We can change the value if needed too.
But if used in an expression before changed, the expression will have
to be rewrite after the changes, if not the old value will still be in use.


}
\end{document}