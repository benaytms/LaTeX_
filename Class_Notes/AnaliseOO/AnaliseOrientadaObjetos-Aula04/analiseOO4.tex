\documentclass{article}
\usepackage[utf8]{inputenc}
\usepackage[a4paper, total={6.5in, 10in}]{geometry}
\usepackage{graphicx}
\usepackage{amsmath}
\usepackage{tgschola}
\usepackage{mathptmx}
\usepackage{fancyhdr}

\newcommand{\minus}{\scalebox{0.8}{$-$}}
\newcommand{\plus}{\scalebox{0.6}{$+$}}
\newcommand{\four}{\scalebox{0.8}{$4$}}

\title{Aula 04 Análise Orientada a Objetos}
\author{Benay Tomas}

\begin{document}

\fontfamily{qcs}\selectfont

\maketitle

\thispagestyle{empty}



\section*{Técnicas de Modelagem da UML}
\begin{itemize}
    \item Diagrama de casos de uso para modelar os requisitos.
    \item Diagrama de atividades para representar o comportamento dos requisitos.
    \item Diagrama de sequência para especificar
    o cenário de funcionalidade dos 
    requisitos funcionais.
\end{itemize}

\subsection*{Transcrição da análise para implementação do projeto}
É adicionado maiores especificações
para os diagramas; 

Tipagem, multiplicidade,
parâmetros dos métodos, estereótipos, etc.

\subsection*{Banco de dados}
São criadas tabelas para cada classe
dos diagramas feitos, onde cada classe
vira uma tabela e cada atributo vira uma coluna
para a tabela.

Essa é uma forma de implementar bancos
de dados na UML.

\section{Diagrama de Pacotes}
Demonstra os elementos dos sistemas agrupados
e organizados em pacotes lógicos.

Agrupa elementos semanticamente
relacionados; Pacotes podem herdar
elementos de outros; Em casos de uso,
agrupa elipses que possuem funções relacionadas.

\begin{center}
\includegraphics*[width=12cm, height=6cm]{packageDiagram.png}
\end{center}

\section{Diagrama de Estrutura Composta}
Diagrama comportamental; Serve para identificar a arquitetura do conjunto
de elementos que interagem entre si num sistema.

\newpage
\thispagestyle{headings}
Exemplo de diagrama de estrutura composta.

\begin{center}
    \includegraphics*[width=12cm, height=8cm]{structureDiagram.png}
\end{center}

\section{Recapitulando}
\begin{itemize}
    \item[] Processo Unificado
    \item[] Diagrama de casos de uso
    \item[] Diagrama de atividades
    \item[] Diagrama de classes
    \item[] Diagrama de sequência
    \item[] Diagrama de comunicação
    \item[] Diagrama de pacotes
    \item[] Diagrama de estrutura composta
    \item[] Diagrama de máquina de estados
    \item[] Diagrama de implantação 
\end{itemize}

\section{Referências}
\begin{enumerate}
    \item Slides da aula 04 (Prof. Vanessa Matias Leite)
\end{enumerate}

\vfill
\begin{center}
    Curitiba

    2024
\end{center}

\end{document}
