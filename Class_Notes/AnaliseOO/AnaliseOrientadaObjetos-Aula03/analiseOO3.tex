\documentclass[12pt, a4paper]{article}
\usepackage{graphicx}
\usepackage{braket}
\usepackage{enumitem}
\usepackage{lmodern}
\usepackage[T1]{fontenc}
\usepackage[margin=1.5cm]{geometry}
\usepackage{textcomp}

\usepackage[
    hidelinks,
    pdfnewwindow=true,
    pdfauthor={Benaytomas},
    pdftitle={Benay Tomas Class Notes},
    pdfpagemode=UseThumbs,
]{hyperref}

\newcommand{\cvsection}[1]{\section*{\rmfamily#1}}
\newcommand{\cvsubsection}[1]{\subsection*{\rmfamily\hspace{1.5em}#1}}
\newcommand{\HL}[1]{\textbf{\color{red}#1}}

\begin{document}
\fontfamily{lmr}\selectfont

\begin{center}

    {\LARGE
        Análise Orientada a Objetos - Aula 03\\
    }
    {\Large
        Benay T. da L. de Carvalho - Universidade Norte do Paraná\\
        RA: 3681235802
    }

\end{center}

\vspace{1cm}

\section{Diagrama Máquinas de Estado}

\indent
\cvsection{Elementos do Diagrama Máquina de Estados}

{\Large
\begin{itemize}
    \item \textbf{Estado inicial/final:} Estado inicial só pode haver um.\\
    Estado final pode ter vários.
    \item \textbf{Estado:} Representa o estado de uma entidade do sistema.
    Contém as cláusulas do que será feito. \\Exemplo: Interface Caixa ATM - exibirSaldoNaTela
    \item \textbf{Transição:} Liga os elementos.
    \item \textbf{Escolha:} Representa um estado condicional.
    \item \textbf{Fork:} Separa uma transição em duas ou mais.
    \item \textbf{Join:} Junta duas ou mais transições em uma.
\end{itemize}
}

\indent
\cvsection{Cláusulas dos Estados}

{\Large

\hspace{1em}Cláusulas representam os tipos de ações em um Estado.

\begin{description}
    \item \textbf{Do:} Ações que o objeto faz pra se manter naquele estado.
    \item \textbf{Entry:} Ações para o objeto assumir novo estado.
    \item \textbf{Exit:} Ações para o objeto sair de estado.
\end{description}

\noindent
\hspace{1em}Em diagramas com muita abstração, não usa-se as cláusulas.
}

\newpage

\indent
\cvsection{Passos para construir um diagrama M.E.}

{\Large
\begin{enumerate}[label=\textbf{\arabic*}º]
    \item Identificar os estados relevantes para os objetos.
    \item Identificar os eventos e transições dos estados.
    \item Verificar se há fatores que influenciam os eventos.
    \item Definir estado inicial e final.
\end{enumerate}
}

\indent
\section{Diagrama de Sequência}

\indent
\cvsection{Elementos do Diagrama de Sequência}

{\Large
\begin{itemize}
    \item \textbf{Lifeline:} Serve para indicar a existência de outros elementos.
    \item \textbf{Ator:} Indica um objeto-pessoa.
    \item \textbf{Objeto:} Entidades que interagem entre si no sistema.
    \item \textbf{Foco de controle:} Linha tracejada onde ocorre a troca de mensagens
    entre objetos.
    \item \textbf{Frames:} Indicam a existência de outro diagrama, serve para melhor
    organizar o diagrama.
    \item \textbf{Mensagem Assíncrona:} Comando que espera resposta.
    \item \textbf{Mensagem Síncrona:} Comando que não espera resposta.
    \item \textbf{Mensagem de Resposta:} Serve para responder as outras mensagens e também
    criar ou destruir elementos. 
\end{itemize}
}

\newpage

\indent
\cvsection{Referências}

{\Large
\hspace{1em}Classificações dos frames
\begin{description}
    \item \textbf{ref:} Faz referência a outro diagrama; Seria o equivalente
    a chamar uma função em programação.
    \item \textbf{alt:} Executará apenas o que for verdadeiro. Pode haver 
    múltiplas opções.
    \item \textbf{opt:} Executará se for verdadeiro. Só haverá uma opção.
    \item \textbf{loop:} Indica um loop
\end{description}
}

\indent
\section{Diagrama de Comunicação}

{\Large
\begin{itemize}
    \item \textbf{Lifeline}
    \item \textbf{Ator}
    \item \textbf{Mensagem}
    \item \textbf{Nota:} Nota para descrever o que for necessário.
    \item \textbf{Vinculo:} Serve para ligar elementos entre si.
    \item \textbf{Vinculo Recursivo:} Serve para ligar um elemento a ele mesmo.
    Como uma função recursiva. Por exemplo, depois de receber o valor de uma senha
    o computador analisa se é a senha correta ele mesmo, sem precisar enviar essa informação
    para outra entidade.
\end{itemize}

\hspace{1em}O Diagrama de Comunicação é muito parecido com o de Sequência
a diferença é que ele é organizado de forma diferente e contém numerações
para indicar a ordem de ocorrência dos eventos.


\vfill
\begin{center}
    Curitiba\\
    2023
\end{center}
}
\end{document}