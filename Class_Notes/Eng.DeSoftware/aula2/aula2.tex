\documentclass{article}
\usepackage[utf8]{inputenc}
\usepackage[a4paper, total={6.5in, 10in}]{geometry}
\usepackage{graphicx}
\usepackage{amsmath}
\usepackage{tgschola}
\usepackage{mathptmx}
\usepackage{fancyhdr}

\newcommand{\minus}{\scalebox{0.8}{$-$}}
\newcommand{\plus}{\scalebox{0.6}{$+$}}
\newcommand{\four}{\scalebox{0.8}{$4$}}

\title{Aula 2 Engenharia de Software}
\author{Benay Tomas}

\begin{document}

\fontfamily{qcs}\selectfont

\maketitle

\thispagestyle{empty}

\textbf{Conteúdos abordados nessa aula}

\textbf{Conceito de qualidade, gestão da qualidade do software e SQA} 

\textbf{(Software Quality Assurance)}

\section{O que é qualidade de Software?}

Qualidade de software se refere ao grau de conformidade
de um software com seus requisitos especificados, e também
com alguns aspectos básicos, dentre esses:
Funcionalidade, desempenho, confiabilidade, usabilidade e manutenção.
Há outros aspectos muito importantes mas esses são os principais.

Em outras palavras, é  a medida em que um software
atende as expectativas e necessidades dos usúarios.

\subsection*{E qual é a importância de um software de qualidade?}

\begin{itemize}
    \item \textbf{Satisfação do cliente:} Software de alta qualidade
    cumpre ou até supera a expectativa do cliente, resultando
    em maior satisfação e lealdade do cliente.
    \item  \textbf{Confiabilidade:} Um software confiável e robusto, 
    funcionando de maneira consistente sem erros ou falhas é crucial.
    \item \textbf{Manuntenção:} Com qualidade de alto nível manutenção
    se torna menos necessária mas de qualquer forma se acontece de precisar,
    um software de qualidade vai permitir que essa manutenção seja feita sem
    maiores problemas. 
    \item \textbf{Eficiência:} Um software de qualidade é eficiente
    em termos de uso de recursos, como tempo de processamento e uso adequado de memória.
    Isso é importante para garantir um bom desempenho do sistema, especialmente
    em grandes volumes de dados ou que exigem resposta rápida.
\end{itemize}

Há muitas outras razões do porquê um software de qualidade
é importante mas esses mencionados já resumem bem.


\section{Gestão da qualidade do software}

Gestão da qualidade de software nada mais é do que
o conjunto de processos e práticas utilizadas para 
garantir que o software atenda aos requisitos de qualidade
esperados. Envolve planejamento, controle e uma melhoria
contína da qualidade ao longo do ciclo de vida do software.

Se trata da garantia que a qualidade do software exista
e continue existindo conforme manutenções e atualizações
são feitas ao dito software.

A importância da gestão se dá em adição ao que vimos
na seção anterior, para que um software de qualidade
se mantenha daquela forma ele deve ser bem gerido.

\section{Software Quality Assurance}
Ou SQA abreviado, se trata de um conjunto
sistemático de atividades realizadas para garantir
que o software desenvolvido atenda aos requisitos de qualidade mencionados
na seção 1. Em suma SQA é o que a gestão utiliza para que seja garantido 
a qualidade de software no sistema.


\newpage
\subsection*{Elementos chave da SQA}

\begin{itemize}
    \item \textbf{Definição e implementação de processos:} Estabelecer
    processos padronizados para desenvolvimento e manutenção de software,
    garantindo consistência e qualidade.
    \item \textbf{Verificar e validar:} SQA assegura que o software atenda às suas especificações
    e objetivos através de testes e revisões.
    \item \textbf{Auditorias e revisões:} Realizar auditorias e revisões
    regulares para identificar e corrigir problemas precocemente.
    \item \textbf{Métricas e medição:} Utilizar métricas para medir a qualidade
    dos processos e produtos, facilitando a melhoria contínua.
\end{itemize}

\subsection*{Importância da SQA}
A importância da garantia de qualidade
reside em assegurar um software o mais próximo possível
de algo impecável e consistente, prevenindo defeitos 
e/ou falhas. SQA reduz custos ao identificar
e corrigir problemas precocemente, evitando
despesas elevadas no começo do projeto. Também
garante conformidade com normas, regulações e requisitos.

\section{Modelos de Qualidade Processos}
Nada mais são do que metodologias que fazem parte do SQA,
onde a intenção deles é garantir que os processos dentro
do software estejam de maior qualidade possível.

Esses modelos são classificados em níveis de maturidade,
que representa quão alto é a qualidade do processo ou do projeto
como todo. Se a empresa possui um software com baixa qualidade
significa que a maturidade da empresa é baixa.

\subsection*{CMMI}
Ou Capability Maturity Model Integration, foi criado
pela Software Engineering Institute. 

O CMMI está dividido em 5 níveis de maturidade
que atestam o grau de evolução em que uma organização
se encontra num determinado momento.
Esses 5 níveis são:

\begin{center}
    \includegraphics*[height=10cm, width=15cm]{cmmi.png}
\end{center}

\newpage
Nível 1: Os processos estão envoltos
em um caos decorrente da não existência de padrões
estabilizados.

Nível 2: O sistema conta com processos
que possuem seus requisitos gerenciados,
porém ainda há falta de medição e controle de diferentes
processos.

Nível 3: Os processo já estão claramente definidos
e são compreendidos dentro da organização. Os procedimentos
se encontram padronizados, além de ser preciso prever sua aplicação
em diferentes projetos.

Nível 4: Ocorre o aumento da previsibilidade
do desempenho de diferentes processos, uma vez que os
mesmos já são controlados quantitativamente.

Nível 5: Os processos estão com ótima qualidade
mas ainda há espaço para melhorias dos processos.

A implantação do CMMI é recomendável para grandes desenvolvedoras
pois 5 níveis de maturidade tendem a não funcionarem bem
com menores empresas. 

Nesse caso o próximo modelo pode funcionar melhor.

\subsection*{MPS.BR}
Também conhecido como Melhoria do Processo de Software Brasileiro,
é uma metodologia voltada à área de desenvolvimento de sistemas que foi criada
por um conjunto de organizações ligadas ao desenvolvimento de software.

Dentro do MPS-BR, enfatiza-se o uso das principais
abordagens internacionais voltadas para a definição, a avaliação e a
melhoria dos processos de software. Isso o torna compatível com as práticas do CMMI.

Também há níveis de maturidade assim como CMMI, 
nesse caso há 7 níveis de maturidade.

\begin{center}
    \includegraphics*[height=10cm, width=14cm]{MPS-BR-maturidade.jpg}
\end{center}

Nível G: Neste ponto inicial deve-se iniciar o
gerenciamento de requisitos e de projetos.

Nível F: Introduz controles de medição, gerência de
configuração, conceitos sobre aquisição e garantia da qualidade.

Nível E: Considera processos como treinamento,
adaptação de processos para gerência de projetos, 
além da preocupação com a melhoria e o controle do processo organizacional.

Nível D: Envolve verificação, validação, além da 
liberação, instalação e integração de produtos, dentre outras atividades.

Nível C: Nesse ponto ocorre o gerenciamento de riscos.

Nível B: Avalia-se o desempenho dos processos,
além da gerência quantitativa dos mesmos.

Nível A: Há a preocupação com questões como inovação
e análise de causas.

\newpage
\thispagestyle{headings}
Cada um desses níveis é composto por um conjunto
de processos de projeto e processos organizacionais,
bem como um conjunto de capacidade de processo.
Esses estão representados abaixo:

\begin{center}
    \includegraphics*[height=10cm, width=15cm]{processos-mps-br.png}
\end{center}

\vspace*{0.5cm}
MPS-BR pode ser considerado uma ótima alternativa ao CMMI
em organizações de médio e pequeno porte.

\section{Referências}
\begin{enumerate}
    \item Maturidade no Desenvolvimento de software: CMMI e MPS-BR, devmedia.com
    \item MPS BR: Níveis de maturidade, promovesolucoes.com
    \item Slides da aula de Engenharia de Software (Prof. Vanessa Matias Leites)
\end{enumerate}

\vfill
\begin{center}
    Curitiba

    2024
\end{center}

\end{document}
