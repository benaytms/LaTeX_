\documentclass[12pt, a4paper]{article}
\usepackage{graphicx}
\usepackage{braket}
\usepackage{enumitem}
\usepackage{indentfirst}
\usepackage[margin=1.5cm]{geometry}

\begin{document}

\begin{center}
    
    {\LARGE
        Sistemas de Computação e de Informação - Aula 02\\
    }
    {\Large
        Benay T. da L. de Carvalho - Universidade Norte do Paraná\\
        RA: 3681235802
    }

\end{center}

\vspace*{1cm}

\section{Fundamentos e Estrutura Geral de um Sistema de Computação}
{\Large

\textbf{Elementos básicos dos computadores}

\begin{enumerate}[label=\textbf{\roman*}]
    \item Unidade de entrada de dados
    \item Processamento de dados
    \item Armazenamento de dados na memória
    \item Saída de dados
\end{enumerate}

\textbf{Hardware}

\begin{enumerate}[label=\textbf{\roman*}]
    \item CPU (Central Processing Unit)
    \item Memória
    \item Dispositivo de entrada
    \item Dispositivo de saída
\end{enumerate}

\textbf{Software}

\begin{enumerate}[label=\textbf{\roman*}]
    \item Sistema Operacional
    \item Aplicativos
    \item Banco de dados
\end{enumerate}

}

\newpage
\vspace*{1cm}

\section{Principais Sistemas Computacionais}

{\Large

\noindent
\textbf{Sistemas de Informação Gerencial}


\textbf{Sistemas de Processamento de Transações (SPT)}
\begin{itemize}
    \item Principal sistema utilizado na maioria dos sistemas da organização
    \item Agrega as principais necessidades dos gerentes operacionais: \\
    Relatórios diários, registro de problemas e necessidades
\end{itemize}

\textbf{Sistemas de Informação Gerencial (SIG)}
\begin{itemize}
    \item Utilizada por gerentes de nível médio para monitoração e controle
    \item Fonte para tomada de decisão tática
    \item Agregam informações completas de um departamento, com integração de outros
\end{itemize}

\textbf{Sistemas de Apoio à Decisão (SAD)}
\begin{itemize}
    \item Utilizado por gerentes e analistas de negócios
    \item Focam problemas únicos e que se alteram com rapidez
    \item Auxiliam em decisões não predefinidas
    \item Tem como base algumas informações de fontes externas
\end{itemize}

\textbf{Ciclo de um Sistema de Informação Gerencial}
\begin{enumerate}[label=\textbf{\Roman*}]
    \item Informação do Mundo Real
    \item Aquisição de dados
    \item Processamento de dados
    \item Análise de dados
    \item Tomada de decisão
\end{enumerate}

\newpage
\vspace*{1cm}

\textbf{Sistema de Relacionamento com o Cliente (CRM)}
\begin{itemize}
    \item Garantia de rentabilidade
    \item Fidelização e retenção de clientes
    \item Armazena informações com intenção de melhorar a relação com o cliente
    \item Têm função básica de atender as necessidades e desejos do cliente
\end{itemize}

\noindent
Há 3 tipos de Sistemas de relacionamento com Cliente:

\begin{itemize}
    \item CRM Operacional
    \item CRM Analítico
    \item CRM Colaborativo 
\end{itemize}

\vfill
\begin{center}
    Curitiba\\
    2023
\end{center}

}
\end{document}