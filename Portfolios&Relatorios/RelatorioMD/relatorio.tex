\documentclass{report}
\usepackage[T1]{fontenc}
\usepackage{lmodern}
\usepackage[a4paper, total={6.5in, 10in}]{geometry}
\usepackage{graphicx}
\usepackage{multirow}
\usepackage{booktabs}
\usepackage{ragged2e}
\usepackage{tabularx}

\fontfamily{lmdh}\selectfont 
\renewcommand\thesection{\Roman{section}}
\renewcommand{\arraystretch}{1.3}


\title{Relatório da Aula Prática $-$ Modelagem de Dados}
\author{Benay T. da L. de Carvalho | $RA: 36812358$}
\date{4 de Maio, 2024}

\begin{document}

\maketitle

\thispagestyle{headings}
\indent
\section{Introdução}
A partir do procedimento da aula prática, foi elaborado um Diagrama
Entidade-Relacionamento utilizando a ferramenta MySQL Workbench. 
E nesse relatório irei explicar o procedimento que deve ser feito para a 
elaboração desse diagrama e o resultado final.

A atividade proposta foi a seguinte, uma biblioteca de uma universidade
realiza empréstimo de suas obras para os alunos da instituição. Tendo isso
em vista, nesse contexto temos:
\begin{itemize}
    \item Aluno: Pessoas que possuem registro na faculdade.
    \item Livro: Publicações físicas com registros passiveis de serem emprestados.
    \item Colaborador: Funcionário da instituição autorizado a fazer empréstimos.
    \item Empréstimo: A ação a ser executada pelo aluno.
\end{itemize}

Com isso, cada uma dessas entidades têm certas características. O aluno terá
RA, nome, email e telefone. O livro terá ISBN, nome, autor e número de páginas.
O colaborador terá CPF, nome, email e cargo. O empréstimo terá ID, data de empréstimo,
data de devolução, ISBN do Livro e CPF do colaborador. Cada um desses itens em maiúsculo
é chamado de chave primária, é o que irá diferenciar cada entidade.

Em seguida irei explicar o processo para produção do diagrama.

\indent
\section{Desenvolvimento}
Antes de começarmos, deverá ser feito o Download da ferramenta MySQL Workbench.

Acesse o site: https://dev.mysql.com/downloads/workbench/ e faça o download.
Após a instalação, abra a ferramenta. Dentro do aplicativo clique em File,
New Model e Dê dois cliques em ``Add Diagram''. Com isso feito, podemos começar.

Primeiro, adicionamos 4 novas tabelas, uma para cada entidade do sistema (Aluno, Livro, Colaborador e Empréstimo).
Depois, colocamos os atributos de cada um levando em consideração
o tipo de cada atributo, os Nomes e Emails estarão em VALCHAR, os valores numéricos em INT
e as datas em DATETIME.
No caso do Empréstimo nós não colocamos ISBN do Livro nem
CPF do colaborador porque esses já são atributos dos outros objetos. 
O que precisamos fazer é colocar a relação de generalização (linha tracejada) 
entre eles para que o Empréstimo herde as chaves primárias do 
Colaborador e do Livro.

Tendo isso feito, precisamos apenas colocar as relações entre essas
entidades e está pronto.

\indent
\section{Resultado}
Esse é o resultado do Diagrama Entidade-Relacionamento feito em MySQL
Workbench.

\begin{center}
    \includegraphics*[width=11cm, height=7cm]{diagramaDER.png}
\end{center}

\newpage
\thispagestyle{headings}

\indent
\section{Conclusão}
Tendo o diagrama pronto, conseguimos ter uma visão do funcionamento de um sistema
básico de uma biblioteca universitária. Com isso conclui-se por meio deste relatório,
a utilidade do Diagrama Entidade-Relacionamento. 

É recomendado a utilização desse diagrama para casos em que 
precisamos visualizar ou desenvolver uma estrutura de um 
banco de dados com uma visão detalhada e expressiva.

\indent
\section{Referências}

\begin{enumerate}
    \item Relationship in ER Diagrams $-$ https://www.datensen.com/blog/data-modeling/relationships-in-er-diagrams/
    \item Slides das Aulas de Modelagem de Dados (Prof. Marco Hisatomi)
\end{enumerate}

\vfill
\begin{center}
    Curitiba

    2024
\end{center}

\end{document}