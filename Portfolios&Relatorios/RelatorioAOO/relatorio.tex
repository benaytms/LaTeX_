\documentclass{report}
\usepackage[T1]{fontenc}
\usepackage{lmodern}
\usepackage[a4paper, total={6.5in, 10in}]{geometry}
\usepackage{graphicx}
\usepackage{multirow}
\usepackage{booktabs}
\usepackage{ragged2e}
\usepackage{tabularx}

\fontfamily{lmdh}\selectfont 
\renewcommand\thesection{\Roman{section}}
\renewcommand{\arraystretch}{1.3}


\title{Relatório da Aula Prática $-$ Análise Orientada a Objetos}
\author{Benay T. da L. de Carvalho | $RA: 36812358$}
\date{3 de Maio, 2024}

\begin{document}

\maketitle

\thispagestyle{headings}
\indent
\section{Introdução}
Seguindo o roteiro para a aula prática, foi feito um diagrama de classes
para simular um sistema de locação de carros, dentro desse diagrama foram
colocadas diferentes classes com diferentes atributos e métodos. Tendo isso
em vista, nesse relatório será abordado a lógica por trás do meu raciocínio 
e os passos que levaram para o resultado final do diagrama mencionado.

\indent
\section{Métodos}
Com base no que foi proposto, foi feito um diagrama de classes
para representar e demonstrar o funcionamento de um sistema em uma 
empresa de locação de carros. 

Os seguintes métodos, atributos e classes foram incluidos:

\begin{center}
    \begin{table}[h]
        \begin{tabular}{  l  p{6cm}  p{6cm} }
            \toprule
\textbf{Classes}      
& \textbf{Atributos}   
& \textbf{Métodos} \\\midrule
Empresa
& Nome, idade da empresa, filial, quantidade de carros e reputação.
& Comprar carros, negociar parceria e pagar funcionário.\\ \hline
Funcionario      
& Nome, idade, numero de identificação, telefone e cargo.                         
& Atender clientes, pegar informação do carro, fornecer formulário e emitir contrato.\\ \hline
Cliente
& Nome completo, idade, endereço, telefone, CPF, RG, foto 3x4 e cópia de documento.
& Preencher formulário, locar carro, devolver carro e pagar alocação. \\\hline
Carro     
& Cor, ano, quilometragem, número da placa e cidade da placa. 
& Esta alocado, data e hora da alocação e data e hora da devolução. \\ \hline
Marca
& Nome, idade da marca, quantidade de carros e reputação.
& Vender carros e fazer parceria. \\ \hline
Modelo
& Nome do modelo, data de lançamento, tipo de combustível e preço de locação.
& Não possui métodos. \\ \hline
Inventário
& Item e quantidade
& Procurar item no inventário, passar informações do item e atualizar estoque. \\
            \bottomrule
        \end{tabular}
    \end{table}
\end{center}

As classes são as abstrações das entidades que serão observadas na vida real.
Quando um objeto de certa classe for instânciado, ele irá herdar todas as características
que sua classe possui. Dito isso as características que uma classe pode ter se resume em
duas, os atributos e métodos.

Os atributos são informações estáticas, eles são valores geralmente imutáveis que
o objeto instânciado terá. Enquanto os métodos são ações que esses objetos podem ter.
Comumente os atributos são escritos constituindo de substantivos e separados por underline.
E os métodos são normalmente escritos com substantivos junto de verbos no infinitivo.

\newpage
\thispagestyle{headings}
\indent
\section{Resultados}
Esse foi o resultado do diagrama de classes feito para simular um sistema de locação
de veículos.

\begin{center}
\includegraphics*[width=16.5cm, height=13cm]{classdiagram_aoo.png}
\end{center}

\indent
\section{Conclusão}
Com isso temos uma visão geral de como um sistema representado por um diagrama de classes
se comporta. Usando esse diagrama podemos entender melhor
sobre as relações e associações entre as entidades do sistema, ao mesmo tempo 
também ter uma visão aprofundada do que consiste cada objeto e 
quais ações os mesmos podem ter.

O diagrama de classes é um dos diagramas mais utilizados da Unified Modelling Language (UML).
Ele pega a ideia do diagrama de casos de uso e adiciona o conceito de orientação a objetos,
resultando em uma modelagem completa e detalhada sem compromenter o entendimento mantendo
o modelo simples e mais abstrato. 

Dessa forma ele se torná perfeito para situações
em que você precisa saber o que cada objeto pode ou não pode fazer.

\newpage
\thispagestyle{headings}
\indent
\section{Referências}

\begin{enumerate}
    \item Fancy Tables in LaTeX $-$ https://tex.stackexchange.com/questions/94032/fancy-tables-in-latex
    \item Slides das aulas de Análise Orientada a Objetos (Prof. Vanessa Matias Leite)
\end{enumerate}

\vfill
\begin{center}
    Curitiba\\
    2024
\end{center}

\end{document}